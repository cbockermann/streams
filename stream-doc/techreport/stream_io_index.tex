%\subsubsection{Package {\ttfamily stream.io}}
\subsection{\label{app:dataStreams}\label{api:stream:io}Data Streams}

Reading data is usually the first step in data processing. The package
{\ttfamily stream.io} provides a set of data stream implementations
for data files/resources in various formats.

All of the streams provided by this package do read from URLs, which
allows reading from files as well as from network URLs such as HTTP
urls or plain input streams (e.g. standard input).

The streams provide an iterative access to the data and use the default
\texttt{DataFactory} for creating data. They do usually share some
common parameters supported by most of the streams such as
\texttt{limit} or \texttt{username} and \texttt{password}.

\subsubsection*{Defining a Stream}
As discussed in Section \ref{sec:designingProcesses}, a stream is
defined within a container using the XML {\ttfamily stream} element,
providing a {\ttfamily url} and {\ttfamily class} attribute which
determines the source to read from and the class that should be used
for reading from that source. In addition, the definition requires a
third attribute {\ttfamily id}, which assigns the stream with a
(unique) identifier. This identifier is then used to reference the
stream as input to a process.

As a simple example, the following XML snippet defines a data stream
that reads data items in CSV format from some file URL:
\begin{figure}[h!]
        \centering
        \begin{lstlisting}{lang=xml}
           <stream  id="csv-data" class="stream.io.CsvStream"
                   url="file:/tmp/example.csv" />
        \end{lstlisting}
        \caption{Defining a CSV stream from a file.}
\end{figure}

\subsubsection*{Streaming Data from various URLs}
The \streams runtime supports a list of different URL schemes which
are provided by all Java virtual machines, e.g. {\ttfamily http} URLs
or {\ttfamily file} URLs. Custom URL schemes can also be registered
within the Java VM. As of this, the \streams runtime additionally
offers a {\ttfamily classpath:} and a {\ttfamily system:} URL scheme.

The {\ttfamily classpath:} URLs can be used to create data streams
that read from resources which are available on the classpath. This is
useful for providing example sources within custom JAR files or the
like. The following example shows how to create a stream that reads
data in JSON format from a resource {\ttfamily example.json} that is
searched for in the default classpath:
\begin{figure}[h!]
        \centering
        \begin{lstlisting}{lang=xml}
           <stream  id="json-stream"  class="stream.io.JSONStream"
                   url="classpath:/example.json" />
        \end{lstlisting}
        \caption{\label{fig:jsonStreamClasspath}Defining a JSON stream from a classpath resource.}
\end{figure}

To support streams that read data from standard input or standard
error, the library provides the {\ttfamily system:} URL schema. This
schema provides access to the system input and error streams and are
useful when piping data to a stream via the command line, e.g. by
running a command like:
\sample{\# cat data.csv | stream.run my-process.xml}
To define a stream that reads from standard input, simply specify
{\ttfamily system:input} as the streams URL as shown in figure
\begin{figure}[h1]
        \centering
        \begin{lstlisting}{lang=xml}
           <stream  id="example"  class="stream.io.CsvStream"
                   url="system:input" />
        \end{lstlisting}
        \caption{\label{fig:csvStreamStdin}Defining a CSV stream that reads data from the system's standard input.}
\end{figure}

\newpage
\subsubsection{ArffStream}

This stream provides access to reading ARFF files and processing them in
a stream based fashion. ARFF is a standard format for data in the
machine learning community with its root in the
(WEKA){[}http://en.wikipedia.org/wiki/Weka\_(machine\_learning){]}
project.

\begin{figure}[h]
\begin{center}
{\renewcommand{\arraystretch}{1.2}
\textsf{
\begin{tabular}{|c|c|p{9cm}|c|} \hline
\textbf{Parameter} & \textbf{Type} & \textbf{Description} & \textbf{Required} \\ \hline  
password & String & The password for the stream URL (see username parameter) & false\\ \hline
prefix & String & An optional prefix string to prepend to all attribute names & false\\ \hline
limit & Long & The maximum number of items that this stream should deliver & false\\ \hline
username & String & The username required to connect to the stream URL (e.g web-user, database user) & false\\ \hline
\end{tabular}
 } 
 } 
\caption{Parameters of processor {\ttfamily ArffStream}}
\end{center}
\end{figure}


\subsubsection{CSVStream}

This data stream source reads simple comma separated values from a
file/url. Each line is split using a separator (regular expression).

Lines starting with a hash character (\texttt{\#}) are regarded to be
headers which define the names of the columns.

The default split expression is \texttt{(;\textbar{},)}, but this can
changed to whatever is required using the \texttt{separator} parameter.

\begin{figure}[h]
\begin{center}
{\renewcommand{\arraystretch}{1.2}
\textsf{
\begin{tabular}{|c|c|p{9cm}|c|} \hline
\textbf{Parameter} & \textbf{Type} & \textbf{Description} & \textbf{Required} \\ \hline  
keys & String[] &  & ? \\ \hline
separator & String &  & true\\ \hline
password & String & The password for the stream URL (see username parameter) & false\\ \hline
prefix & String & An optional prefix string to prepend to all attribute names & false\\ \hline
limit & Long & The maximum number of items that this stream should deliver & false\\ \hline
username & String & The username required to connect to the stream URL (e.g web-user, database user) & false\\ \hline
\end{tabular}
 } 
 } 
\caption{Parameters of processor {\ttfamily CsvStream}}
\end{center}
\end{figure}


\subsubsection{\label{api:stream:io:JSONStream}Class {\ttfamily stream.io.JSONStream}}

This data stream reads JSON objects from the source (file/url) and
returns the corresponding Data items. The stream implementation expects
each line of the file/url to provide a single object in JSON format.

\begin{figure}[h]
\begin{center}{\footnotesize
{\renewcommand{\arraystretch}{1.4}
\textsf{
\begin{tabular}{|c|c|p{9cm}|c|} \hline
\textbf{Parameter} & \textbf{Type} & \textbf{Description} & \textbf{Required} \\ \hline  
{\ttfamily id } & String &  & ? \\ \hline
{\ttfamily password } & String & The password for the stream URL (see username parameter) & false\\ \hline
{\ttfamily prefix } & String & An optional prefix string to prepend to all attribute names & false\\ \hline
{\ttfamily limit } & Long & The maximum number of items that this stream should deliver & false\\ \hline
{\ttfamily username } & String & The username required to connect to the stream URL (e.g web-user, database user) & false\\ \hline
\end{tabular}
 } 
 } 
 } 
\caption{Parameters of class {\ttfamily stream.io.JSONStream}}
\end{center}
\end{figure}

\subsubsection{LineStream}

This is a very simple stream that just reads from a URL line-by-line.
The content of the line is stored in the attribute determined by the
\texttt{key} parameter. By default the key \texttt{LINE} is used.

It also supports the specification of a simple format string that can be
used to create a generic parser to populate additional fields of the
data item read from the stream.

The parser format is:

\begin{verbatim}
  %(IP) [%(DATE)] "%(URL)"
\end{verbatim}
This will create a parser that is able to read line in the format

\begin{verbatim}
  127.0.0.1 [2012/03/14 12:03:48 +0100] "http://example.com/index.html"
\end{verbatim}
The outcoming data item will have the attribute \texttt{IP} set to
\texttt{127.0.0.1} and the \texttt{DATE} attribute set to
\texttt{2012/03/14 12:03:48 +0100}. The \texttt{URL} attribute will be
set to \texttt{http://example.com/index.html}. In addition, the
\texttt{LINE} attribute will contain the complete line string.

\begin{figure}[h]
\begin{center}{\footnotesize
{\renewcommand{\arraystretch}{1.4}
\textsf{
\begin{tabular}{|c|c|p{9cm}|c|} \hline
\textbf{Parameter} & \textbf{Type} & \textbf{Description} & \textbf{Required} \\ \hline  
{\ttfamily key } & String &  & false\\ \hline
{\ttfamily format } & String & The format how to parse each line. Elements like {\ttfamily \%(KEY)} will be detected and automatically populated in the resulting items. & false\\ \hline
{\ttfamily id } & String &  & ? \\ \hline
{\ttfamily password } & String & The password for the stream URL (see username parameter) & false\\ \hline
{\ttfamily prefix } & String & An optional prefix string to prepend to all attribute names & false\\ \hline
{\ttfamily limit } & Long & The maximum number of items that this stream should deliver & false\\ \hline
{\ttfamily username } & String & The username required to connect to the stream URL (e.g web-user, database user) & false\\ \hline
\end{tabular}
 } 
 } 
 } 
\caption{Parameters of class {\ttfamily stream.io.LineStream}}
\end{center}
\end{figure}

\subsubsection{ProcessStream}

This processor executes an external process (programm/script) that
produces data and writes that data to standard output. This can be used
to use external programs that can read files and stream those files in
any of the formats provided by the stream API.

The default format for external processes is expected to be CSV.

In the following example, the Unix command \texttt{cat} is used as an
example, producing lines of some CSV file:

\begin{verbatim}
   <Stream class="stream.io.ProcessStream"
           command="/bin/cat /tmp/test.csv"
           format="stream.io.CsvStream" />
\end{verbatim}
The process is started at initialization time and the output will be
read from standard input.

\begin{figure}[h]
\begin{center}
{\renewcommand{\arraystretch}{1.2}
\textsf{
\begin{tabular}{|c|c|p{9cm}|c|} \hline
\textbf{Parameter} & \textbf{Type} & \textbf{Description} & \textbf{Required} \\ \hline  
format & String & The format of the input (standard input), defaults to CSV & true\\ \hline
command & String & The command to execute. This command will be spawned and is assumed to output data to standard output. & true\\ \hline
\end{tabular}
 } 
 } 
\caption{Parameters of processor {\ttfamily ProcessStream}}
\end{center}
\end{figure}

\StreamSection{SQLStream}

This class implements a DataStream that reads items from a SQL database
table. The class requires a \texttt{jdbc} URL string, a username and
password as well as a \texttt{select} parameter that will select the
data from the database.

The following XML snippet demonstrates the definition of a SQL stream
from a database table called \texttt{TEST\_TABLE}:
\begin{figure}[h!]
     \begin{lstlisting}[language=XML]
         <stream class="stream.io.SQLStream"
                   url="jdbc:mysql://localhost:3306/TestDB"
              username="SA" password=""
                select="SELECT * FROM TEST_TABLE" />
     \end{lstlisting}
     \caption{\label{fig:exampleSQLStream}Example SQL streams, reading from a database.}
\end{figure}
The database connection is established using the user \texttt{SA} and no
password (empty string). The above example connects to a MySQL database.

As the SQL database drivers are not part of the streams library, you
will need to provide the database driver library for your database on
the class path.

\begin{table}[h]
\begin{center}{\footnotesize
{\renewcommand{\arraystretch}{1.4}
\textsf{
\begin{tabular}{|c|c|p{9cm}|c|} \hline
\textbf{Parameter} & \textbf{Type} & \textbf{Description} & \textbf{Required} \\ \hline  
{\ttfamily id } & String & The ID of the stream with which it is assicated to proceses.  & true \\ \hline
{\ttfamily url } & String & The JDBC database url to connect to. & true\\ \hline
{\ttfamily select } & String & The select statement to select items from the database. & true\\ \hline
{\ttfamily password } & String & The password for the stream URL (see username parameter) & false\\ \hline
{\ttfamily prefix } & String & An optional prefix string to prepend to all attribute names. & false\\ \hline
{\ttfamily limit } & Long & The maximum number of items that this stream should deliver. & false\\ \hline
{\ttfamily username } & String & The username required to connect to the stream URL (e.g web-user, database user) & false\\ \hline
\end{tabular}
 } 
 } 
 } 
\caption{Parameters of class {\ttfamily stream.io.SQLStream}.}
\end{center}
\end{table}

\subsubsection{SvmLightStream}

This stream implementation provides a data stream for the SVMlight
format. The SVMlight format is a simple \texttt{key:value} format for
compact storage of high dimensional sparse labeled data. It is a line
oriented format where each line is laid out as shown in Figure
\ref{fig:sampleSvmLight}. The keys are usually indexes, but this
stream implementation also supports string keys.
\begin{figure}[h!]
\sample{
-1.0 4:3.3 10:0.342 44:9.834 \# some comment
}
\caption{\label{fig:sampleSvmLight}A sample line of a SVMLight file.}
\end{figure}
The \texttt{\#} character starts a comment that can be provided to
each line.

\begin{table}[h]
\begin{center}{\footnotesize
{\renewcommand{\arraystretch}{1.4}
\textsf{
\begin{tabular}{|c|c|p{9cm}|c|} \hline
\textbf{Parameter} & \textbf{Type} & \textbf{Description} & \textbf{Required} \\ \hline  
{\ttfamily sparseKey } & String &  & ? \\ \hline
{\ttfamily id } & String & The ID of this string for associating it with processes. & true\\ \hline
{\ttfamily password } & String & The password for the stream URL (see username parameter) & false\\ \hline
{\ttfamily prefix } & String & An optional prefix string to prepend to all attribute names. & false\\ \hline
{\ttfamily limit } & Long & The maximum number of items that this stream should deliver. & false\\ \hline
{\ttfamily username } & String & The username required to connect to the stream URL (e.g web-user, database user) & false\\ \hline
\end{tabular}
 } 
 } 
 } 
\caption{Parameters of class {\ttfamily stream.io.SvmLightStream}.}
\end{center}
\end{table}

\paragraph{TimeStream}

This is a very simple stream that emits a single data item upon every
read. The data item contains a single attribute \texttt{@timestamp} that
contains the current timestamp (time in milliseconds).

The name of the attribute can be changed with the \texttt{key}
parameter, e.g.~to obtain the timestamp in attribute \texttt{@clock}:

\begin{verbatim}
  <Stream class="stream.io.TimeStream" key="@clock" />
\end{verbatim}


\begin{figure}[h]
\begin{center}
{\renewcommand{\arraystretch}{1.2}
\textsf{
\begin{tabular}{|c|c|p{9cm}|c|} \hline
\textbf{Parameter} & \textbf{Type} & \textbf{Description} & \textbf{Required} \\ \hline  
interval & String &  & ? \\ \hline
id & String &  & ? \\ \hline
password & String & The password for the stream URL (see username parameter) & false\\ \hline
prefix & String & An optional prefix string to prepend to all attribute names & false\\ \hline
limit & Long & The maximum number of items that this stream should deliver & false\\ \hline
username & String & The username required to connect to the stream URL (e.g web-user, database user) & false\\ \hline
\end{tabular}
 } 
 } 
\caption{Parameters of processor {\ttfamily TimeStream}}
\end{center}
\end{figure}



%\paragraph{BlockingQueue}

The \emph{BlockingQueue} provides a simple DataStream that items can be
enqueued into and read from. This allows inter-process communication
between multiple active processes to be designed using data items as
messages.

\emph{TODO:} Write details about the queuing behavior!

\begin{figure}[h]
\begin{center}
{\renewcommand{\arraystretch}{1.2}
\textsf{
\begin{tabular}{|c|c|p{9cm}|c|} \hline
\textbf{Parameter} & \textbf{Type} & \textbf{Description} & \textbf{Required} \\ \hline  
size & int &  & ? \\ \hline
id & String &  & ? \\ \hline
password & String & The password for the stream URL (see username parameter) & false\\ \hline
prefix & String & An optional prefix string to prepend to all attribute names & false\\ \hline
limit & Long & The maximum number of items that this stream should deliver & false\\ \hline
username & String & The username required to connect to the stream URL (e.g web-user, database user) & false\\ \hline
\end{tabular}
 } 
 } 
\caption{Parameters of processor {\ttfamily BlockingQueue}}
\end{center}
\end{figure}

%\paragraph{CsvUpload}

This processor simply uploads all processed items to a remote (HTTP)
URL. The upload consists of a single POST request for each item. The
POST request contains a header line and the data line as produced by the
\emph{CsvWriter}.


%\paragraph{CsvWriter}

This processor appends all processed data items to a file in CSV format.
The processor either adds all keys of the items or only a set of
previous defined keys.

As first line, the writer emits a header line with a comma separated
list of column names. This line is prepended with a \texttt{\#}
character.

The processor supports the creation of files with varying numbers of
keys/attributes. If an item is processed with a different (larger)
number of keys and the set of keys has not been defined in the
\texttt{keys} parameter, a new header will be inserted into the file,
signaling the header for the next items to be written.

By default the writer uses \texttt{,} as separator, which can be changed
by the \texttt{separator} parameter.

The following example shows a \emph{CsvWriter} writing to
\texttt{/tmp/test.csv} using the \texttt{;} as separator. Only keys
\texttt{@id} and \texttt{name} will be written:

\begin{verbatim}
  <CsvWriter url="file:/tmp/test.csv" keys="@id,name"
             separator=";" />
\end{verbatim}


\begin{figure}[h]
\begin{center}
{\renewcommand{\arraystretch}{1.2}
\textsf{
\begin{tabular}{|c|c|p{9cm}|c|} \hline
\textbf{Parameter} & \textbf{Type} & \textbf{Description} & \textbf{Required} \\ \hline  
keys & String[] & The attributes to write out, leave blank to write out all attributes. & false\\ \hline
url & String & The url to write to. & true\\ \hline
separator & String & The separator to separate columns, usually ',' & false\\ \hline
attributeFilter & String &  & ? \\ \hline
condition & String & The condition parameter allows to specify a boolean expression that is matched against each item. The processor only processes items matching that expression. & false\\ \hline
\end{tabular}
 } 
 } 
\caption{Parameters of processor {\ttfamily CsvWriter}}
\end{center}
\end{figure}

%\subsubsection{JSONWriter}

This processor writes the processed items to a file/url. The output is
written in JSON format, where a single line of JSON output is produced
for each processed item.

The result can be parsed/read using the
(JSONStream){[}JSONStream.html{]}.

\begin{figure}[h]
\begin{center}
{\renewcommand{\arraystretch}{1.2}
\textsf{
\begin{tabular}{|c|c|p{9cm}|c|} \hline
\textbf{Parameter} & \textbf{Type} & \textbf{Description} & \textbf{Required} \\ \hline  
keys & String[] & The attributes to write out, leave blank to write out all attributes. & false\\ \hline
url & String & The url to write to. & true\\ \hline
separator & String & The separator to separate columns, usually ',' & false\\ \hline
attributeFilter & String &  & ? \\ \hline
condition & String & The condition parameter allows to specify a boolean expression that is matched against each item. The processor only processes items matching that expression. & false\\ \hline
\end{tabular}
 } 
 } 
\caption{Parameters of processor {\ttfamily JSONWriter}}
\end{center}
\end{figure}

%\subsubsection{LineWriter}

This processor simply writes out items to a file in text format. The
format of the file is by default a single line for each item. Any
occurrences of new lines in the values of each item are escaped by
backslash escaping.

The following example will create a single line for each item starting
with some constant string, followed by the value of the items
\texttt{@id} attribute, a constant string \texttt{-\textgreater{}} and
the items \texttt{name} attribute:

\begin{verbatim}
  <LineWriter format="UserId: %{data.@id} -> %{data.name}"
              file="/tmp/example.out.txt"/>
\end{verbatim}


\begin{figure}[h]
\begin{center}
{\renewcommand{\arraystretch}{1.2}
\textsf{
\begin{tabular}{|c|c|p{9cm}|c|} \hline
\textbf{Parameter} & \textbf{Type} & \textbf{Description} & \textbf{Required} \\ \hline  
format & String & The format string, containing macros that are expanded for each item & true\\ \hline
file & File & Name of the file to write to. & true\\ \hline
append & boolean & Denotes whether to append to existing files or create a new file at container startup. & false\\ \hline
escapeNewlines & boolean & Whether to escape newlines contained in the attributes or not. & false\\ \hline
\end{tabular}
 } 
 } 
\caption{Parameters of processor {\ttfamily LineWriter}}
\end{center}
\end{figure}

%\subsubsection{ListDataStream}

This class implements the DataStream interface and can be used to create
a stream instance based on a collection of already available data items.

The purpose of this class is to programmatically provide a data stream
implementation, e.g.~for testing.


%\subsubsection{SQLWriter}

This processor inserts processed items into a SQL database table. At
initialization time, it checks for existence of the table and creates
the table based on the keys of the first item if the table does not
exist.

If the table exists beforehand, the table schema will be extracted and
only keys with corresponding table columns will be inserted.

The following example shows the configuration of the SQLWriter to insert
the keys \texttt{@id} and \texttt{attr1}, \texttt{attr2} into the table
\texttt{DATA}:

\begin{verbatim}
<SQLWriter keys="@id,attr1,attr2"
           url="jdbc:hsqldb:file:/tmp/test.db"
           username="SA" password=""
           table="DATA" />
\end{verbatim}
The parameters \texttt{url}, \texttt{username} and \texttt{password}
define the connection to the database, whereas the \texttt{table}
parameter defines the table into which data is to be inserted.

\subparagraph{Dropping existing Tables}

The \emph{SQLWriter} also allows to drop existing tables at
initialization time by specifying the parameter \texttt{dropTable} as
\texttt{true}.

\begin{figure}[h]
\begin{center}
{\renewcommand{\arraystretch}{1.2}
\textsf{
\begin{tabular}{|c|c|p{9cm}|c|} \hline
\textbf{Parameter} & \textbf{Type} & \textbf{Description} & \textbf{Required} \\ \hline  
table & String & The database table to insert items into. & true\\ \hline
keys & String[] & A list of attributes to insert (columns), empty string for all attributes. & false\\ \hline
dropTable & boolean &  & false\\ \hline
password & String & The password used to connect to the database. & false\\ \hline
username & String & The username used to connect to the database. & false\\ \hline
url & String & The JDBC database url to connect to. & true\\ \hline
\end{tabular}
 } 
 } 
\caption{Parameters of processor {\ttfamily SQLWriter}}
\end{center}
\end{figure}

%\subsubsection{SvmLightStream}

This stream implementation provides a data stream for the SVMlight
format. The SVMlight format is a simple \texttt{key:value} format for
compact storage of high dimensional sparse labeled data. It is a line
oriented format where each line is laid out as shown in Figure
\ref{fig:sampleSvmLight}. The keys are usually indexes, but this
stream implementation also supports string keys.
\begin{figure}[h!]
\sample{
-1.0 4:3.3 10:0.342 44:9.834 \# some comment
}
\caption{\label{fig:sampleSvmLight}A sample line of a SVMLight file.}
\end{figure}
The \texttt{\#} character starts a comment that can be provided to
each line.

\begin{table}[h]
\begin{center}{\footnotesize
{\renewcommand{\arraystretch}{1.4}
\textsf{
\begin{tabular}{|c|c|p{9cm}|c|} \hline
\textbf{Parameter} & \textbf{Type} & \textbf{Description} & \textbf{Required} \\ \hline  
{\ttfamily sparseKey } & String &  & ? \\ \hline
{\ttfamily id } & String & The ID of this string for associating it with processes. & true\\ \hline
{\ttfamily password } & String & The password for the stream URL (see username parameter) & false\\ \hline
{\ttfamily prefix } & String & An optional prefix string to prepend to all attribute names. & false\\ \hline
{\ttfamily limit } & Long & The maximum number of items that this stream should deliver. & false\\ \hline
{\ttfamily username } & String & The username required to connect to the stream URL (e.g web-user, database user) & false\\ \hline
\end{tabular}
 } 
 } 
 } 
\caption{Parameters of class {\ttfamily stream.io.SvmLightStream}.}
\end{center}
\end{table}

%\paragraph{SvmLightWriter}

The SvmLightWriter is a simple processor implementation that writes all
processed items to a file or URL using the SVMlight format. The format
supports sparse vectors and a single label attribute.

The format is described in (SvmLightStream){[}SvmLightStream.html{]}.

\begin{figure}[h]
\begin{center}
{\renewcommand{\arraystretch}{1.2}
\textsf{
\begin{tabular}{|c|c|p{9cm}|c|} \hline
\textbf{Parameter} & \textbf{Type} & \textbf{Description} & \textbf{Required} \\ \hline  
includeAnnotations & boolean &  & ? \\ \hline
keys & String[] & The attributes to write out, leave blank to write out all attributes. & false\\ \hline
url & String & The url to write to. & true\\ \hline
separator & String & The separator to separate columns, usually ',' & false\\ \hline
attributeFilter & String &  & ? \\ \hline
condition & String & The condition parameter allows to specify a boolean expression that is matched against each item. The processor only processes items matching that expression. & false\\ \hline
\end{tabular}
 } 
 } 
\caption{Parameters of processor {\ttfamily SvmLightWriter}}
\end{center}
\end{figure}

%\paragraph{TimeStream}

This is a very simple stream that emits a single data item upon every
read. The data item contains a single attribute \texttt{@timestamp} that
contains the current timestamp (time in milliseconds).

The name of the attribute can be changed with the \texttt{key}
parameter, e.g.~to obtain the timestamp in attribute \texttt{@clock}:

\begin{verbatim}
  <Stream class="stream.io.TimeStream" key="@clock" />
\end{verbatim}


\begin{figure}[h]
\begin{center}
{\renewcommand{\arraystretch}{1.2}
\textsf{
\begin{tabular}{|c|c|p{9cm}|c|} \hline
\textbf{Parameter} & \textbf{Type} & \textbf{Description} & \textbf{Required} \\ \hline  
interval & String &  & ? \\ \hline
id & String &  & ? \\ \hline
password & String & The password for the stream URL (see username parameter) & false\\ \hline
prefix & String & An optional prefix string to prepend to all attribute names & false\\ \hline
limit & Long & The maximum number of items that this stream should deliver & false\\ \hline
username & String & The username required to connect to the stream URL (e.g web-user, database user) & false\\ \hline
\end{tabular}
 } 
 } 
\caption{Parameters of processor {\ttfamily TimeStream}}
\end{center}
\end{figure}

%\paragraph{TreeStream}

This stream is a line oriented stream that expects each line to provide
a tree structure in the format

\begin{verbatim}
  ( ROOT ( A1 ( A1.1 ) ( A1.2 ) ) ( A2 ) )
\end{verbatim}


\begin{figure}[h]
\begin{center}
{\renewcommand{\arraystretch}{1.2}
\textsf{
\begin{tabular}{|c|c|p{9cm}|c|} \hline
\textbf{Parameter} & \textbf{Type} & \textbf{Description} & \textbf{Required} \\ \hline  
id & String &  & ? \\ \hline
treeAttribute & String &  & ? \\ \hline
\end{tabular}
 } 
 } 
\caption{Parameters of processor {\ttfamily TreeStream}}
\end{center}
\end{figure}

