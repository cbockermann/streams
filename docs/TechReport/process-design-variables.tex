\subsection{\label{sec:processVariables}Parameterising Containers}
The general structure of container definitions described in Section
\ref{sec:processLayout} allows for the definition of compute graphs
and adding processors and parameters.
For a convenient parameterization, the \streams framework supports the
global definition of properties and includes a intuitive variable
expansion, following the syntax of well known tools like Ant and
Maven.

Variables are specified using the {\ttfamily \$} symbol and curly
bracket wrapped around the property name, e.g. {\ttfamily
  \$\{myVar\}}. This directly allows to access the Java VM system
properties within the container definition. Undefined variables
simple resolve to the empty string.

\begin{figure}[h!]
  \centering
  \begin{lstlisting}
     <container>

         <!-- define property 'baseUrl' using the system property 'user.home'   -->
         <property name="baseUrl" value="file:${user.home}/data/FACT" />

         <stream id="factData" class="fact.io.FactEventStream"
                url="${baseUrl}/example-data.gz" />

         <process input="factData">
            <!-- process the data  -->
         </process>
     </container>
  \end{lstlisting}
  \caption{\label{fig:propertyExample}A container definition using simple variables.}
\end{figure}

As the variable expansion includes the Java system properties, containers
can easily be provided with variables by setting properties when starting
the Java system. The following commands starts the \streams runtime with
a container definition and adds addition variables:

\vspace{1ex}\hspace{2ex}\sample{java -DbaseUrl="/tmp" -cp stream-runner.jar container.xml}

Variables can be used anywhere in the XML attributes, the variables of
a container are expanded at startup time. Therefore any changes of the
variables after the container has been started will not affect the
configuration.
