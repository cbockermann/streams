\StreamSection{LineStream}

This class provides a very flexible stream implementations that
essentially reads from a URL line-by-line. The content of the complete
line is stored in the attribute determined by the \texttt{key}
parameter. By default the key \texttt{LINE} is used.

It also supports the specification of a simple format/grammar string
that can be used to create a generic parser to populate additional
fields of the data item read from the stream.

The grammar is a string containing {\ttfamily \%(name)} elements,
where {\ttfamily name} is the name of the attribute that should be
created at that specific portion of the line. An example, for such
a simple grammar is given as follows:
\begin{verbatim}
  %(IP) [%(DATE)] "%(URL)"
\end{verbatim}
The {\ttfamily \%(name)} elements are extracted from the grammar and
all remaining elements in between are regarded as boundary strings that
separate the elements.

The simple grammar above will create a parser that is able to read
lines in the format of the following:
\begin{verbatim}
  127.0.0.1 [2012/03/14 12:03:48 +0100] "http://example.com/index.html"
\end{verbatim}


The outcoming data item will have four attributes {\ttfamily LINE},
{\ttfamily IP}, {\ttfamily DATE} and {\ttfamily URL}. The attribute
\texttt{IP} set to \texttt{127.0.0.1} and the \texttt{DATE} attribute
set to \texttt{2012/03/14 12:03:48 +0100}. The \texttt{URL} attribute
will be set to \texttt{http://example.com/index.html}. The
\texttt{LINE} attribute will contain the complete line string.

\begin{table}[h]
\begin{center}{\footnotesize
{\renewcommand{\arraystretch}{1.4}
\textsf{
\begin{tabular}{|c|c|p{9cm}|c|} \hline
\textbf{Parameter} & \textbf{Type} & \textbf{Description} & \textbf{Required} \\ \hline  
{\ttfamily id } & String & The ID of the stream with which it is assicated to proceses.  & true \\ \hline
{\ttfamily key } & String & The name of the attribute holding the complete line, defaults to {\ttfamily LINE}. & false\\ \hline
{\ttfamily format } & String & The format how to parse each line. Elements like {\ttfamily \%(KEY)} will be detected and automatically populated in the resulting items. & false\\ \hline
{\ttfamily password } & String & The password for the stream URL (see username parameter) & false\\ \hline
{\ttfamily prefix } & String & An optional prefix string to prepend to all attribute names & false\\ \hline
{\ttfamily limit } & Long & The maximum number of items that this stream should deliver & false\\ \hline
{\ttfamily username } & String & The username required to connect to the stream URL (e.g web-user, database user) & false\\ \hline
\end{tabular}
 } 
 } 
 } 
\caption{Parameters of class {\ttfamily stream.io.LineStream}}
\end{center}
\end{table}
\afterpage{\clearpage}