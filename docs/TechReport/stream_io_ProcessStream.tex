\newpage
\StreamSection{ProcessStream}
This processor executes an external process (programm/script) that
produces data and writes that data to standard output. This can be used
to use external programs that can read files and stream those files in
any of the formats provided by the stream API.

The default format for external processes is expected to be CSV.
In the following example, the Unix command \texttt{cat} is used as an
example, producing lines of some CSV file:
\begin{figure}[h!]
  \centering
  \begin{lstlisting}{lang=xml}
     <stream   class="stream.io.ProcessStream"
             command="/bin/cat /tmp/test.csv"
              format="stream.io.CsvStream" />
  \end{lstlisting}
\end{figure}

The process is started at initialization time and the output will be
read from standard input.

\begin{table}[h]
\begin{center}{\footnotesize
{\renewcommand{\arraystretch}{1.4}
\textsf{
\begin{tabular}{|c|c|p{9cm}|c|} \hline
\textbf{Parameter} & \textbf{Type} & \textbf{Description} & \textbf{Required} \\ \hline  
{\ttfamily id } & String & The ID of the stream with which it is assicated to proceses.  & true \\ \hline
{\ttfamily format } & String & The format of the input (standard input), defaults to CSV & true\\ \hline
{\ttfamily command } & String & The command to execute. This command will be spawned and is assumed to output data to standard output. & true\\ \hline
\end{tabular}
 } 
 } 
 } 
\caption{Parameters of class {\ttfamily stream.io.ProcessStream}.}
\end{center}
\end{table}
\afterpage{\clearpage}
