\Processor{WithKeys}

This processor is a processor list that executes one or more inner
processors. It creates a copy of the current data item with all
attributes matching the list of specified keys. Then all nested
processors are applied to that copy and the copy is merged back into
the original data item.

If any of the nested data items returns \emph{null}, this processor will
also return \emph{null}.

The \texttt{keys} parameter of this processor allows for specifying a
comma separated list of keys and key-patterns using simple wildcards
\texttt{*} and \texttt{?} as shown in Figure \ref{fig:withKeys}. If
the {\ttfamily keys} parameter is not provided, then the inner
processors will be provided with a complete copy of the current data
item.

\begin{figure}[h!]
\begin{lstlisting}{lang=xml}
      <process ...>
          <WithKeys keys="x1,user:*,!user:id">
             <PrintData />
          </WithKeys>
      </process>
\end{lstlisting}
\caption{\label{fig:withKeys}Selects only attribute {\ttfamily x1},
  all attributes starting with {\ttfamily user:} but not attribute
  {\ttfamily user:id} and executes the {\ttfamily PrintData} processor
  for this selection of attributes.}
\end{figure}


\begin{table}[h]
\begin{center}{\footnotesize
{\renewcommand{\arraystretch}{1.4}
\textsf{
\begin{tabular}{|c|c|p{9cm}|c|} \hline
\textbf{Parameter} & \textbf{Type} & \textbf{Description} & \textbf{Required} \\ \hline  
{\ttfamily keys } & String[] & A list of filter keys selecting the attributes that should be provided to the inner processors. & false\\ \hline
{\ttfamily merge } & Boolean & Indicates whether the outcome of the inner processors should be merged into the input data item, defaults to true. & false\\ \hline
\end{tabular}
 } 
 } 
 } 
\caption{Parameters of class {\ttfamily stream.data.WithKeys}.}
\end{center}
\end{table}
