\subsubsection{Delay}

This simple processor puts a delay into the data processing. The delay
can be specified in various units with the simple time format being
specified like:

\begin{verbatim}
  40ms
\end{verbatim}
for specifying a delay of 40 milliseconds. Other units for \texttt{day},
\texttt{hour}, \texttt{minute} and so on work as well.

The units can also be combined as in \texttt{1 second 30ms}.

\begin{figure}[h]
\begin{center}
{\renewcommand{\arraystretch}{1.2}
\textsf{
\begin{tabular}{|c|c|p{9cm}|c|} \hline
\textbf{Parameter} & \textbf{Type} & \textbf{Description} & \textbf{Required} \\ \hline  
time & String &  & true\\ \hline
condition & String & The condition parameter allows to specify a boolean expression that is matched against each item. The processor only processes items matching that expression. & false\\ \hline
\end{tabular}
 } 
 } 
\caption{Parameters of processor {\ttfamily Delay}}
\end{center}
\end{figure}
