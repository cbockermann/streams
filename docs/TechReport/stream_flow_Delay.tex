\Processor{Delay}

This simple processor puts a delay into the data processing. The delay
can be specified in various units with the simple time format being
specified like ``{\ttfamily 40ms}'' for specifying a delay of 40
milliseconds. Other units for \texttt{day}, \texttt{hour},
\texttt{minute} and so on work as well.

The units can also be combined as in \texttt{1 second 30ms}.

\begin{table}[h]
\centering
{\footnotesize
{\renewcommand{\arraystretch}{1.4}
\textsf{
\begin{tabular}{|c|c|p{9cm}|c|} \hline
\textbf{Parameter} & \textbf{Type} & \textbf{Description} & \textbf{Required} \\ \hline  
{\ttfamily time } & String & The time that the data flow should be delayed. & true\\ \hline
{\ttfamily condition } & String & The condition parameter allows to specify a boolean expression that is matched against each item. The processor only processes items matching that expression. & false\\ \hline
\end{tabular}
 } 
 }
}
\caption{Parameters of class {\ttfamily stream.flow.Delay}.}
\end{table}
