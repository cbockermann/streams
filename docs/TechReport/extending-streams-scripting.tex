\subsection{\label{sec:scripting}Using Scripting Languages in \streams}
Scripting languages provide a convenient way to integrate ad-hoc
functionality into stream processes. Based on the Java Scripting
Engine that is provided within the Java virtual machine, the streams
library includes support for several scripting languages, most notably
the JavaScript language.

Additional scripting languages are being supported by the
ScriptingEngine interfaces of the Java virtual machine. This requires
the corresponding Java implementations (Java archives) to be available
on the classpath when starting the \streams runtime.

Currently the following scripting languages are supported:
\begin{itemize}
   \item JavaScript (built into the Java VM)
   \item JRuby (requires jruby-library in classpath).
\end{itemize}
Further support for integrating additional languages like Python is
planned.\baustelle


\subsubsection{\label{sec:javascript}Using JavaScript for Processing}
The JavaScript language has been part of the Java API for some
time. The \streams framework provides a simple {\ttfamily JavaScript}
processor, that can be used to run JavaScript functions on data items
as shown in Figure \ref{fig:javascriptExample}.

\begin{figure}[h!]
  \centering
  \begin{lstlisting}[language=XML]
       <container>
          ...
          <process input="...">
              <!--  Execute a process(data) function defined in the
                    specified JavaScript file     -->
              <JavaScript file="/path/to/myScript.js" />
          </process>
       </container>
  \end{lstlisting}
  \caption{\label{fig:javascriptExample}The {\ttfamily JavaScript} processor applies process() functions defined in JavaScript.}
\end{figure}


Within the JavaScript environment, the data items are accessible at
{\ttfamily data}. Figure \ref{fig:javascriptProcessor} shows an
example for the JavaScript code which implements a processor within
the file {\ttfamily myScript.js}.

\begin{figure}[h!]
  \centering
  \begin{lstlisting}[language=JavaScript,showspaces=false,showstringspaces=false]
       function process(data){
          var id = data.get( "@id" );
          if( id != null ){
             println( "ID of item is: " + id );
          }
          return data;
        }
  \end{lstlisting}
  \caption{\label{fig:javascriptProcessor}JavaScript code that implements a processor.}
\end{figure}