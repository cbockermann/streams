\subsubsection{BinaryLabels}

This processor transforms items into binary labeled items with labels
with values \emph{-1.0} or \emph{+1.0}. The processor provides several
parameters that can be used for adjusting the strategy on how to map
labels.

By default the label attribute is expected to be in key \texttt{@label}.
This can be changed to the correct key name using the \texttt{label}
parameter.

\subparagraph{Mapping String labels}

In the case of labels of string type, the processor needs to be provided
with the string value of the positive class using the \texttt{positive}
parameter. Items with the label key set to the specified value for
\texttt{positive} will be flagged with a label \emph{+1.0}, all other
will be marked as \emph{-1.0}.

The following example sets up a \emph{BinaryLabels} processor mapping
all items with a label \texttt{@label} and value \texttt{SPAM} to
\emph{+1.0}. Items with any other label value are marked as \emph{-1.0}:

\begin{verbatim}
  &lt;BinaryLabels label="@label" positive="SPAM" /&gt;
\end{verbatim}
If no value for \texttt{positive} is specified, the first value for that
label key is regarded as the positive class value.

\subparagraph{Mapping Numberic Labels}

If the labels are of numberic type, the mapping can be done with a
simple threshold comparison. For this, the processor provides a
\texttt{threshold} parameter.

In the following example, items with a \texttt{@label} value higher than
\emph{0.5} are mapped to class \emph{+1.0}, all others are mapped to
class \emph{-1.0}:

\begin{verbatim}
  &lt;BinaryLabels label="@label" threshold="0.5" /&gt;
\end{verbatim}
If no threshold is specified, the default threshold of \emph{0.0} is
used.

\begin{figure}[h]
\begin{center}
{\renewcommand{\arraystretch}{1.2}
\textsf{
\begin{tabular}{|c|c|p{9cm}|c|} \hline
\textbf{Parameter} & \textbf{Type} & \textbf{Description} & \textbf{Required} \\ \hline  
threshold & Double &  & ? \\ \hline
label & String &  & ? \\ \hline
positive & String &  & ? \\ \hline
\end{tabular}
 } 
 } 
\caption{Parameters of processor {\ttfamily BinaryLabels}}
\end{center}
\end{figure}
