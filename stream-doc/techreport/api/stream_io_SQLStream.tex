\subsubsection{SQLStream}

This class implements a DataStream that reads items from a SQL database
table. The class requires a \texttt{jdbc} URL string, a username and
password as well as a \texttt{select} parameter that will select the
data from the database.

The following XML snippet demonstrates the definition of a SQL stream
from a database table called \texttt{TEST\_TABLE}:

\begin{verbatim}
 <Stream class="stream.io.SQLStream"
         url="jdbc:mysql://localhost:3306/TestDB"
         username="SA" password=""
         select="SELECT * FROM TEST_TABLE" />
\end{verbatim}
The database connection is established using the user \texttt{SA} and no
password (empty string). The above example connects to a MySQL database.

As the SQL database drivers are not part of the streams library, you
will need to provide the database driver library for your database on
the class path.

\begin{figure}[h]
\begin{center}
{\renewcommand{\arraystretch}{1.2}
\textsf{
\begin{tabular}{|c|c|p{9cm}|c|} \hline
\textbf{Parameter} & \textbf{Type} & \textbf{Description} & \textbf{Required} \\ \hline  
url & String & The JDBC database url to connect to. & true\\ \hline
select & String & The select statement to select items from the database. & true\\ \hline
password & String & The password for the stream URL (see username parameter) & false\\ \hline
prefix & String & An optional prefix string to prepend to all attribute names & false\\ \hline
limit & Long & The maximum number of items that this stream should deliver & false\\ \hline
username & String & The username required to connect to the stream URL (e.g web-user, database user) & false\\ \hline
\end{tabular}
 } 
 } 
\caption{Parameters of processor {\ttfamily SQLStream}}
\end{center}
\end{figure}
