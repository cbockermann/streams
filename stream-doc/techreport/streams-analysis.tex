\clearpage
\section{\label{sec:machineLearning}Machine Learning with Continuous Data}
As the large volumes of data are merely handable with automatic
processing of that data, they are far away from being inspected
manually. On the other hand gaining insight from that data is the key
problem in various application domains.

Machine learning and data mining has put forth a plethora of
techniques and algorithms for pattern recognition, learning of
prediction model or clustering that all aim at exactly that key
problem: knowledge discover from data.

For the setting of continuous data, various algorithms have been
proposed which solve basic tasks inherent to the knowledge discovery
process as well as complex methods that allow for training classifiers
or finding clusters on steady streams of data. In this section we will
give an overview of how machine learning algorithms are embedded into
the \streams framework using a simple Naive Bayes classifier as example.
% that have been
%implemented within the \streams framework as well as the integration
%of an existing library, namely the MOA \cite{moa}, which provides a
%rich set of state of the art online learning algorithms. 

We first give an overview of the data representation that is used for
learning from the {\em data items} that represent the basic data format
of {\em streams}. Following that we outline how learning algorithms are
embedded into the framework of continuous processes. Here we provide an
example for online classification and the computation of statistics.

Based on the embedding of online learning schemes for classification,
we show the integration of the MOA library into the \streams
framework, which allows for directly using the set of existing
classifiers for learning (Section \ref{sec:moa}).

%Following that, we outline the existing online learning implementations
%provided by the {\em streams-analysis} package in Section \ref{sec:streamsAnalysis}.
%A large set of online learning methods is already provided by the {\em MOA}
%library, which is directly integrated into the {\em streams-analysis} package.
%We give details on the integration of {\em MOA} in Section \ref{sec:moa}.

The evaluation of online learning often requires large amounts of data.
In Section \ref{sec:syntheticData} we show how to generate synthetic data
streams for testing online learning algorithms.

%After that overview in Section \ref{sec:onlineLearning}, we discuss
%the problem of the {\em online application} of machine learning models
%which have either been trained online of offline, and then are used to
%make real-life predictions on streaming data. This will be covered in
%Section \ref{sec:onlineApplication}.

%\subsubsection*{Notation used}
%Within this section, we will denote each data item $d_i$ obtained from
%a stream as a tuple $d_i = (d_{i,1},\ldots,d_{i,k}) \subset
%M_1\times\ldots\times M_k$, where each $M_i$ might refer to some
%domain, e.g. $M_i \subseteq \mathbb{R}$ or an arbitrary set. Further
%we will refer to the index $i$ as the index of the item $d_i$ in a
%data stream $D$, i.e.
%\begin{displaymath}
%  D = \langle d_0,d_1,\ldots \rangle.
%\end{displaymath}


%%
%%
%%
\subsection{\label{sec:onlineLearning}Online Learning from Data Streams}
The general definitions of learning tasks in online learning do not
differ from the traditional objectives. Supervised learning such as
classification or regression tasks rely on a source of training data
to build models that can then be applied to new data for prediction.

% ... \baustelle
Learning from unbounded and continuous data poses tight challenges to
the designer of machine learning algorithms. Even simple basic
building blocks like the computation of a median or minima/maxima
values that might be required in a learning algorithms tend to become
difficult.

\subsubsection{\label{sec:dataExamples}Learning from Data Items}
Online learning algorithms usually require a data representation
similar to batch learning methods. Typically instances or examples
used for learning are tuples of some real-valued or finite space.

As an example, the task of (binary) classification can be stated as
estimating a function $\hat{f}$ that best approximates a true
(unknown) distribution of instances $(x,y)$ where $x\in M^r$ and $y
\in \{ -1,1 \}$. Usually features are encoded such that $M=\mathbb{R}$
in many application domains.

In the \streams framework we encode each of these tuples as data
items by defining a key $k_i$ for each dimension of $M^p$ and a
special key for the label $y$. By convention, special keys are
prefixed with an {\ttfamily @} character. These special keys are
expected to be ignored as attributes by any learning algorithm.
Figure \ref{fig:exampleItem} shows an instance fo learning that
is represented by a data item.

\begin{figure}[h!]
  \centering
$(x,y) \stackrel{\mbox{\scriptsize e.g.}}{=} (0.3,0.57,\ldots,0.413,-1) \ \ \rightarrow\ \  $ {\footnotesize \begin{tabular}{c|c|c|c|c}
{\bf \textsf{Key}} & {\ttfamily x1} & $\cdots$ & {\ttfamily xp} & {\ttfamily @label} \\ \hline
{\bf \textsf{Value}} & 0.3 & $\cdots$ & 0.413 & -1
%{\bf Value} & {\ttfamily x1} & 0.4 \\ 
%$\vdots$ & $\vdots$ \\ 
%{\ttfamily xp} & 0.413 \\
%{\ttfamily y} & -1 \\
\end{tabular}}
  \caption{\label{fig:exampleItem}Data item representation of an instance for learning, key/value table transposed for brevity.}
\end{figure}

As the attributes may hold any {\ttfamily Serializable} values, a
proper pre-processing might be required for applying learning
algorithms, e.g. if these algorithms cannot handle arbitrary data
types. Such preprocessing is for example a String-to-Number conversion
(provided by the {\ttfamily ParseDouble} processor). The \streams
core classes provide a wide number of preprocessing processors.

\subsubsection*{Filtering Attributes}
Sometimes it is desirable to train a classifier only on a subset of
the features/attributes that are contained in the data. The {\ttfamily WithKeys}
processor, allows for the execution of nested processors on filtered
data items. As an example, the XML snippet in Figure \ref{fig:exampleWithKeys}
shows the data preprocessing to apply online learning to the famous Iris
data set with only two of the attributes being selected.

\begin{figure}[h!]
  \centering
  \begin{lstlisting}[language=XML]
    ...
    <process input="iris">
       
        <!-- Rename the "class" attribute to "@label" as by convention
             a learner expects the label in attribute "@label"        -->
        <Rename from="class" to="@label" />

        <!-- select two attributes and the label from the
             data items and apply the inner processors />
        <WithKeys keys="att1,att2,@label">

            <!-- parse the attributes att1 and att2 to Double values -->
            <ParseDouble keys="att1,att2" />

            <!-- feed the data item to a naive bayes classifier for training -->
            <stream.classifier.NaiveBayes id="myNaiveBayes" />
        </WithKeys>

    </process>
  \end{lstlisting}
  \caption{\label{fig:exampleWithKeys}Example XML for training a
    classifier on a subset of attributes.}
\end{figure}


%\subsubsection*{Approximating Distributions}
%As a simple example, the {\em NaiveBayes}\cite{NB} classifier often
%offers a adequate prediction performance in a lot of application
%domains. Starting with the independence assumption of attributes, it
%maximizes computes the class probabilities given the observed
%attributes of a set of training instances. The base rule of bayes is
%given in equation (\ref{eqn:naiveBayes}):
%\begin{eqnarray}
%  P(c | f_1,\ldots,f_n ) = \frac{P(c)\cdot P(f_1,\ldots,f_n|c)}{P(f_1,\ldots,f_n)}.\label{eqn:naiveBayes}
%\end{eqnarray}

%Assuming a fixed set $C = \{c_1,\ldots,c_l\}$ of observable classes, we
%can easily approximate $P(c)$ by counting the occurences for each
%$c_i$ in the observed stream. For pure numerical attributes $f_i$,
%generally a gaussian normal distribution is assumed, such that the
%factors $P(f_1,\ldots,f_n|c)$ and $P(f_1,\ldots,f_n)$ can be derived
%by estimating the mean and average for each attribute $f_i$.

%The setting gets a bit more complicated if the $f_i$ are nominal
%values, such as variable strings from an unbounded domain such as
%URLs, i.e. $M_i \cong \mathbb{N}$. In this case we cannot simply
%derive a probability for each instance of an attribute as this would
%require counting an unbounded set of strings, which clearly violates
%the stream processing contraints mentioned in Section
%\ref{sec:streamSetting}.

%%
%% How are classifiers embedded into the streams framework?
%% How can they be used?
%%
%\subsubsection{Embedding Classifiers in \streams}


%%
%% Which classifiers/clusterers/etc. are available?
%%
%\subsubsection{Available Online Learning Algorithms}

%\subsubsection*{Online Statistics (Counting, Quantiles)}
%
%\subsubsection*{Online Classifier}

%%
%%
%%
\subsection{\label{sec:streamsClassifiers}Classifiers in \streams}
The abstract functionality provided by classifiers from the perspective
of the \streams framework encapsulates two functions:
\begin{enumerate}
  \item {\bf Training:} Incorporate new data items into a prediction model.
  \item {\bf Application:} Provide a prediction for a data item based on the current model.
\end{enumerate}
These two tasks are mapped onto two different aspects of the compute
graph that builds the basis for the streaming processes. The {\em
  training} is considered to be part of the general data flow,
i.e. data items are processed by classifiers and will be used to
enhance the prediction model provided by the classifier.

The model {\em application}, i.e. the prediction based on the current
model of the classifier, is regarded as an {\em anytime service} that
is provided by the classifier. This service provides a {\ttfamily
  predict(Data)} function that is expected to return the prediction of
the classifier.

\begin{figure}[h!]
  \centering
  \begin{lstlisting}[language=Java]
     public interface Classifier extends Service {
         /**
          * This method returns a simple prediction based on the given
          * data item. The prediction is a general serializable value.
          */
         public Serializable predict( Data item );
     }
  \end{lstlisting}
  \caption{\label{fig:classifierService}The {\ttfamily Classifier}
    service interface that needs to be implemented by classifiers in
    the \streams framework. The return type of the {\ttfamily predict}
    method might be a number, e.g. for regression or a String, Integer
    or similar for a classification task.}
\end{figure}



\subsection{\label{sec:moa}Integrating MOA}

MOA is a software package for online learning tasks. It provides a
large set of clustering and classifier implementations suited for
online learning. Its main intend is to serve as an environment for
evaluating online algorithms.

The \streams framework provides the {\ttfamily stream-analysis}
artifact, which includes MOA and allows for integrating MOA
classifiers directly into standard stream processes. This is achieved
by wrapping the data item processed in the \streams framework into
instances required for MOA. Additionally, a generic class wraps all
the MOA classifier implementations into a processor that implements
the {\ttfamily Classifier} interface. MOA classifiers will be
automatically discovered on the classpath using Java's reflection API
and will be added to the processors available.

The following example XML snippet shows the use of the Naive Bayes
implementation of MOA within a \streams container. The example defines
a standard test-then-train process.


\begin{figure}[h!]
  \centering
  \begin{lstlisting}[language=XML]
      <container>
           <stream id="stream" class="stream.io.CsvStream"
                   url="classpath:/multi-golf.csv.gz" limit="100"/>

           <process input="stream">
                <RenameKey from="play" to="@label" />
        
                <!-- add  @prediction:NB based on the classifier "NB"  -->
                <stream.learner.AddPrediction classifier="NB" />

                <!-- compute the loss for all attributes starting with @prediction:
                     and add a corresponding @error: attribute with the loss   -->
                <stream.learner.evaluation.PredictionError />

                <!-- incorporate the data item in to the model (learning)  -->
                <moa.classifiers.bayes.NaiveBayes id="NB"/>

                <!-- incrementally group the @error:NB   -->
                <stream.statistics.Sum keys="@error:NB" />
           </process>
      </container>
  \end{lstlisting}
  \caption{\label{fig:testThenTraing}Test-then-train evaluation of the
    MOA Naive Bayes classifier using the {\ttfamily AddPrediction}
    processor and the {\ttfamily Sum} processor to sum up the
    prediction error.}
\end{figure}

%\subsubsection{The {\ttfamily moa} packages}
%The {\ttfamily stream-analysis} module of the \streams library uses a
%simple wrapper approach to integrate the MOA classes into the streams
%framework. All implementations of MOA are mapped to their default Java
%package, i.e.
%%
%
%\begin{figure}[h!]
%  \centering
%  \begin{lstlisting}[language=XML]
%   ...
%     <process input="..">
%
%         <moa.classifiers.bayes.NaiveBayes />
%
%     </process>
%   ...
%  \end{lstlisting}
%  \caption{\label{fig:moaClassifierXML}}
%\end{figure}
%
%The options used in MOA are directly mapped to XML element attributes.


\subsection{\label{sec:syntheticData}Synthetic Data Stream Generator}
Testing online algorithms often requires a large amount of data that
matches a known distribution or can be designed such that specific
test-cases can be created for algorithms.

The \streams core package already defines a set of streams for random
data generation. In Combination with the concept of MultiStreams
this can easily be used to create tailored data streams.


\subsubsection{Example: A Gaussian Stream}
The {\ttfamily stream.generator.GaussianStream} class implements a
data stream that generates an unlimited sequence of normal distributed
data. The default setup focuses on a single attribute with a mean of
0.0 and a standard deviation of 1.0:

\begin{figure}[h!]
  \centering
  \begin{lstlisting}[language=XML]
     <stream id="gauss" class="stream.generator.GaussianStream" />
  \end{lstlisting}
\end{figure}

Using the {\ttfamily attributes} parameter allows to specify the mean and
standard deviation of one or more attributes:

\begin{figure}[h!]
  \centering
  \begin{lstlisting}[language=XML]
     <stream id="gauss-2" class="stream.generator.GaussianStream"
             attributes="0.0,1.0,2.0,0.25,8.5,2.75" />
  \end{lstlisting}
\end{figure}

The {\ttfamily gauss-2} stream above produces a sequence of data items
each of which holds attributes {\ttfamily x1}, {\ttfamily x2} and
{\ttfamily x3} based on the following distributions:

\begin{table}[h!]
  \centering
  \begin{tabular}{c|c|c} \hline
    {\bf Attribute} & {\bf Mean} & {\bf Standard Deviation} \\ \hline \hline
    {\ttfamily x1} & 0.0 & 1.0 \\ \hline
    {\ttfamily x2} & 2.0 & 0.25 \\ \hline
    {\ttfamily x3} & 8.5 & 2.75 \\ \hline
  \end{tabular}
\end{table}

The attributes are named {\ttfamily x1}, {\ttfamily x2} and {\ttfamily
  x3} but can be named according to a preset using the {\ttfamily keys}
parameter of the {\ttfamily GaussianStream} class:

\begin{figure}[h!]
  \centering
  \begin{lstlisting}[language=XML]
     <stream id="gauss-2" class="stream.generator.GaussianStream"
             attributes="0.0,1.0,2.0,0.25,8.5,2.75"
             keys="A,B,C" />
  \end{lstlisting}
\end{figure}

\subsubsection{Example: A cluster data-stream}
The stream {\ttfamily gauss-2} from above will create a sequence of
data items which are centered around (0.0,2.0,8.5) in a 3-dimensional
vector space.

By combining the concept of {\em Multistreams} with the gaussian
streams, we can easily define a stream that has multiple clusters with
pre-defined centers. The {\ttfamily RandomMultiStream} class is of big
use, here: It allows for randomly picking a substream upon reading
each item. The picks are uniformly distributed over all substreams.

The following definition specifies a stream with data items of 4
clusters with cluster centers (0.0,0.0), (1.0,1.0), (2.0,2.0) and
(3.0,3.0):

\begin{figure}[h!]
  \centering
  \begin{lstlisting}[language=XML]
    <stream id="clusters" class="stream.io.multi.RandomMultiStream">

        <stream id="cluster-1" class="stream.generator.GaussianStream"
                attributes="1.0,0.0,1.0,0.0" />

        <stream id="cluster-2" class="stream.generator.GaussianStream"
                attributes="2.0,0.0,2.0,0.0" />

        <stream id="cluster-3" class="stream.generator.GaussianStream"
                attributes="3.0,0.0,3.0,0.0" />

        <stream id="cluster-4" class="stream.generator.GaussianStream"
                attributes="4.0,0.0,4.0,0.0" />
    </stream>
  \end{lstlisting}
\end{figure}

\subsubsection{Imbalanced Distributions}
In some cases a unified distribution among the sub-streams is not what
is required. The {\ttfamily weights} parameters lets you define a
weight for each substream, resulting in a finer control of the
stream. As an example, the {\ttfamily weights} parameter can be used
to create a stream with a slight fraction of outlier data items:

\begin{figure}[h!]
  \centering
  \begin{lstlisting}[language=XML]
    <stream id="myStream" class="stream.io.multi.RandomMultiStream"
            weights="0.99,0.01">

        <stream id="normal" class="stream.generator.GaussianStream"
                attributes="1.0,0.0,1.0,0.0" />

        <stream id="outlier" class="stream.generator.GaussianStream"
                attributes="2.0,0.0,2.0,0.0" />
    </stream>
  \end{lstlisting}
\end{figure}
In this example, approximately 1\% of the data items is drawn from the
outlier stream, whereas the majority is picked from the ``normal''
stream.


%%%
%% What is online model application? When is it required?
%% What methods are provided? RapidMiner-Models?
%%
\subsection{\label{sec:onlineApplication}Model Application on Data Streams}

As previously outlined in Section \ref{sec:onlineLearning}, each
classifier provides a service that can be used to access or use its
model. Such services can for example be a {\em PredictionService},
which provides a prediction function for a data item. The former
section mainly focused on the online training of such classifiers,
whereas in this part, we deal with the application of such
classifier/learning output to online data streams.

There are two key aspects we want to discuss here, namely the use of
classifier models for the evaluation of online learning algorithms in
Section \ref{sec:evalOnline} and the general benefit of applying
models that may even have been trained offline in the setting of
continuous data streams in Section \ref{sec:applyOnline}.


%%
%% Raise some questions on "why" and "when" online model application
%% is required, touch "concept drift" problems,...
%%
%% keywords: monitoring, static models, expert knowledge?
%%

%%
%% A simple but important use of online model application is the 
%% evaluation of classifiers on online data streams.
%%
%% This section should give an example walk-through for test-then-train
%% and demonstrate an example container that evaluates a classifier (MOA?)
%% on a data stream.
%%
\subsubsection{\label{sec:evalOnline}Evaluating Classifiers on a Stream}




%%
%% Section about the need to create a model on offline data and
%% using that model on dynamic continuous data streams. 
%% Outline:
%%   - high-level (short) introduction to a use-case
%%   - outlining the offline training of models
%%   - incorporation of the models into the streams library
%%   - maybe a more detailed example to sum up (plus container definiton in the appendix?)
%%
\subsubsection{\label{sec:applyOnline}Learning Offline, Predicting Online}


