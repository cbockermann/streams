\subsubsection{RenameKey}

This processor simply renames a single attribute. The attribute value
will be removed from the data item and added with the new key,
overwriting any existing entry with that new key.

Example:

\begin{verbatim}
   <RenameKey key="myKey" to="myNewName" />
\end{verbatim}


\begin{figure}[h]
\begin{center}
{\renewcommand{\arraystretch}{1.2}
\textsf{
\begin{tabular}{|c|c|p{9cm}|c|} \hline
\textbf{Parameter} & \textbf{Type} & \textbf{Description} & \textbf{Required} \\ \hline  
from & String & The old name of the key. & true\\ \hline
to & String & The new name of the key. & true\\ \hline
condition & String & The condition parameter allows to specify a boolean expression that is matched against each item. The processor only processes items matching that expression. & false\\ \hline
\end{tabular}
 } 
 } 
\caption{Parameters of processor {\ttfamily RenameKey}}
\end{center}
\end{figure}
