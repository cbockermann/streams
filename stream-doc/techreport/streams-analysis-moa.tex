\subsection{\label{sec:moa}Integrating MOA}

MOA is a software package for online learning tasks. It provides a
large set of clustering and classifier implementations suited for
online learning. Its main intend is to serve as an environment for
evaluating online algorithms.

The \streams framework provides the {\ttfamily stream-analysis}
artifact, which includes MOA and allows for integrating MOA
classifiers directly into standard stream processes.

The following example XML snippet shows the use of the Naive Bayes
implementation of MOA within a \streams container. The example defines
a standard test-then-train process.


\begin{figure}[h!]
  \centering
  \begin{lstlisting}[language=XML]
      <container>
           <stream id="stream" class="stream.io.CsvStream"
                   url="classpath:/multi-golf.csv.gz" limit="100"/>

           <process input="stream">
                <RenameKey from="play" to="@label" />
        
                <stream.learner.Prediction learner="NB" />

                <stream.learner.evaluation.PredictionError />

                <moa.classifiers.bayes.NaiveBayes id="NB"/>

                <stream.statistics.Sum keys="@error:NB" />
           </process>
      </container>
  \end{lstlisting}
  \caption{\label{fig:testThenTraing}}
\end{figure}

\subsubsection{The {\ttfamily moa} packages}
The {\ttfamily stream-analysis} module of the \streams library uses a
simple wrapper approach to integrate the MOA classes into the streams
framework. All implementations of MOA are mapped to their default Java
package, i.e.


\begin{figure}[h!]
  \centering
  \begin{lstlisting}[language=XML]
   ...
     <process input="..">

         <moa.classifiers.bayes.NaiveBayes />

     </process>
   ...
  \end{lstlisting}
  \caption{\label{fig:moaClassifierXML}}
\end{figure}

The options used in MOA are directly mapped to XML element attributes.
