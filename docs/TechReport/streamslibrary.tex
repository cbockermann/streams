\section{\label{sec:streamsLibrary}The \streams Library}
The \streams library provides a set of classes and interfaces for the
elements defined in Section \ref{sec:abstraction}, which allows for
implementing custom streams and processors. In addition it provides
basic classes for reading, writing and processing data, e.g. from CSV
files, SVMlight formatted data or by reading streams from an SQL
database. The library consists of three packages:
\begin{enumerate}
\item \textsf{stream-api} -- a small collection of interfaces and classes
  representing the conceptual elements outlined above.
\item \textsf{stream-core} -- several implementations of I/O streams,
  processors, etc. which are of general use.
\item \textsf{stream-runtime} -- a light-weighted execution environment
  that allows to define streaming processes in XML.
\end{enumerate}
To a large extend we focused on developing the \streams API as simple
as possible using standard data structures and following design
patterns and conventions like JavaBeans \cite{javabeans} or techniques
like dependency injection \cite{dependencyInjection} found in well
established frameworks such as the Spring Framework
\cite{springframework} or Google Guice \cite{guice}.

\subsection{Data Items and Processors}
In the \textsf{stream-api} data items are represented by the
{\ttfamily stream.Data} interface, which itself is a plain Java
{\ttfamily Map} with keys of string type and any serializable objects
as values. Maps support our objective to use versatile data structures
that are available and well understood in any language
(e.g. dictionaries in Python or Ruby) and do provide the
self-contained property. The serialization requirement allows data
items to be transferred over network connections, required for running
stream processes in distributed environments.

A data stream is provided by the interface {\ttfamily
  stream.io.DataStream} and basically provides a single {\ttfamily
  readNext()} method returning the next data item of the stream. In
general, the data stream implementations in the \streams library
require a URL or a Java {\ttfamily InputStream} object to read from.
This allows creating streams to read from file, network resources or
from external data generating processes by reading from standard
input.

The processor elements are defined by a simple interface {\ttfamily
  stream.Processor} that requires a single method to be implemented as
shown in Figure \ref{fig:processorInterface}.

\begin{figure}
{\footnotesize
  \begin{lstlisting}
    public interface Processor {
       /**  Method called for each item to be processed.   */
       public Data process( Data item );
    }
  \end{lstlisting}
}
  \caption{\label{fig:processorInterface}Definition of the basic
    processor interface, required to implement custom processors
    within the \streams library.}
\end{figure}
%\begin{figure}[h!]
%
%\begin{verbatim}
%    public interface Processor {
%       /**
%        *  Method called for each item to be processed.
%        */
%       public Data process( Data item );
%    }
%\end{verbatim}
%}
%  \caption{\label{fig:processorInterface}Definition of the basic
%    processor interface, required to implement custom processors
%    within the \streams library.}
%\end{figure}

\subsubsection*{Parameters via JavaBeans}
Following the JavaBeans convention, processors are required to provide
a no-args constructor and may use parameters by simply providing
{\ttfamily get}- and {\ttfamily set}-methods. The example processor
shown in Figure \ref{fig:alertProcessor} (see Appendix) outputs an
alert message for every item that does not provide a ({\em key},{\em
  value}) pair for a user defined key name.  This simple beans
convention allows for automatically registering RapidMiner operators
and their corresponding parameter types. This is provided by the {\em
  RapidMiner-Beans} library.

%\vspace{6cm}

\subsection{Anytime Services}
For implementing the anytime paradigm, the \streams library provides a
{\ttfamily Service} interface and a dynamic naming service which
allows for registering and obtaining services or references to
services. This works similarly to the standard RMI naming services
included in Java, but tries to abstract from a specific
implementation.

The anytime services within the \streams library are implemented by
extending the {\ttfamily Service} interface and defining any method
that shall be provided in an anytime manner. As an example, the
{\ttfamily PredictionService} is implemented by all online learning
algorithms, which defines a simple {\ttfamily predict} method as shown in
Figure \ref{fig:predictionService}. As soon as a processor that
implements a {\ttfamily Service} interface is added to an experiment,
it is automatically registered within the naming service.

\begin{figure}[h!]
\footnotesize
\begin{lstlisting}
   public interface PredictionService extends Service {
       /** Returns the prediction for an item based 
        *  on the current model                      */
       public Serializable predict( Data item );
   }
\end{lstlisting}
\caption{\label{fig:predictionService}A simple {\ttfamily
    PredictionService} that as is provided by all online learning
  algorithms that support classification.}
\end{figure}


%\subsection{Events, Examples and Processors}
%Within RapidMiner, the most fundamental data structure is provided by
%an {\ttfamily IOObject}. Any object exchanged between two operators
%has to implement the {\ttfamily IOObject} interface. From a data
%analysis perspective, these objects typically are sets of examples,
%prediction models, preprocessing models or analysis results such as
%performance vectors.
%
%As the stream processing is mostly concerned with the handling of
%single pieces of data, i.e. single {\em examples} in the RapidMiner
%sense, we denote with {\em data item} the most atomic element of data
%obtained and processed within a stream. A data item is a set of ({\em
%  key},{\em value}) pairs, each of which corresponds to an attribute
%and its value.

\subsection{A light-weight \streams Runtime}
For rapid prototyping and development purposes, the \streams library
implements a small multi-threaded runtime environment, which allows to
define stream processes using a simple XML document. The
interpretation and structure of this XML is very similar to the
notation known from frameworks like Spring \cite{springframework}. A
sample XML process definition of the {\em test-then-train} use case is
provided in the Appendix (Figure \ref{fig:exampleContainer}).
% defines a single stream and a process that
%implements the test-then-train setup mentioned earlier.

The services defined within the \streams API are exported via a naming
service. The default naming service uses a local RMI registry, which
allows for accessing services such as prediction services, or
processors providing meta-data statistics (average, minimum, maximum,
top-k elements) while the processes are running.
