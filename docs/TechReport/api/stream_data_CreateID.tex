\subsubsection{CreateID}

This processor simply adds an incremental identifier to each processed
data item. By default, this identifier is stored as feature
\texttt{@id}, but can be used with any other name as well.

The following example creates a processor adding IDs with name
\texttt{@uid}:

\begin{verbatim}
 <CreateID key="@uid" />
\end{verbatim}
IDs are numbered starting from 0, but can also start at arbitrary
integer values:

\begin{verbatim}
 <CreateID key="@uid" start="10" />
\end{verbatim}


\begin{figure}[h]
\begin{center}
{\renewcommand{\arraystretch}{1.2}
\textsf{
\begin{tabular}{|c|c|p{9cm}|c|} \hline
\textbf{Parameter} & \textbf{Type} & \textbf{Description} & \textbf{Required} \\ \hline  
key & String &  & false\\ \hline
start & Long &  & false\\ \hline
\end{tabular}
 } 
 } 
\caption{Parameters of processor {\ttfamily CreateID}}
\end{center}
\end{figure}
