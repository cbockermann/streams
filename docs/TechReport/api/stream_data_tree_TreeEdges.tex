\paragraph{TreeEdges}

This processor acts upon all trees found in the item or the specified
tree attributes if the parameter \texttt{keys} has been specified.

The processor will add a new attribute
\texttt{key:edge{[}A-\textgreater{}B{]}} for each edge found in the
tree. This attribute will contain the number of times this edge is found
in the tree.

For example, the following definition

\begin{verbatim}
   &lt;TreeEdges keys="@sql:tree" /&gt;
\end{verbatim}
will add attributes in the following scheme:

\begin{verbatim}
   sql:tree[ROOT->Select] = 1.0
   sql:tree[Select->ResultList] = 1.0
   sql:tree[Select->FromList] = 1.0
   sql:tree[Select->WhereClause] = 1.0
\end{verbatim}
Please note, that a possible leading \texttt{@} character is removed
from the key before creating the new attribute keys. Usually the trees
are regarded as special attributes whereas the tree edge attributes are
plain features in the machine learning sense and not special anymore.

