\subsubsection{NGrams}

This parser processor will create n-grams from a specified attribute of
the processed item and will add all the n-grams and their frequency to
the item. By default the processor creates n-grams of length 3.

To not overwrite any existing keys, the n-gram frequencies can be
prefixed with a user-defined string using the \texttt{prefix} parameter.

The following example shows an \emph{NGram} processor that will create
5-grams of the string found in key \texttt{text} and add their frequency
to the items with a prefix of \texttt{5gram}:

\begin{verbatim}
  <NGrams n="5" key="text" prefix="5gram" />
\end{verbatim}


\begin{figure}[h]
\begin{center}
{\renewcommand{\arraystretch}{1.2}
\textsf{
\begin{tabular}{|c|c|p{9cm}|c|} \hline
\textbf{Parameter} & \textbf{Type} & \textbf{Description} & \textbf{Required} \\ \hline  
prefix & String & An optional prefix that is to be prepended for all n-gram names before these are added to the data item & false\\ \hline
key & String & The attribute which is to be split into n-grams & true\\ \hline
n & Integer & The length of the n-grams that are to be created & true\\ \hline
\end{tabular}
 } 
 } 
\caption{Parameters of processor {\ttfamily NGrams}}
\end{center}
\end{figure}
