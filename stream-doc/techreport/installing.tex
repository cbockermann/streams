\section{The \streams Runtime}
Along with the \streams API, that is provided for implementing custom
streams or processors, the \streams framework provides a runtime
environment for running stream containers.

\subsection{Installing the \streams Runtime on Debian/RedHat}
For Debian and RPM based systems, there exists a package repository,
that provides Debian and RPM packages that can easily be installed
using the system's package managers. A step-by-step guide for setting
up the package manager on Debian and Ubuntu systems is provided in
Section \ref{sec:installDeb}. Instructions for RedHat based systems
such as RedHat, CentOS or Scientific Linux are provided in
\ref{sec:installRPM}.


\subsubsection{\label{sec:gpgKey}Signatures for Packages}
The repositories and all packages within the repository are
cryptographically signed with a GPG key with ID {\ttfamily 0x13443F4A}
to ensure their consistency. The key is available at

\sample{http://download.jwall.org/software.gpg}

The key is associated with the following information:
\sample{User ID: jwall.org Software Repository <software@jwall.org>\newline
Fingerprint: 175C 915F 51CA 8AA2 387B  E3E8 48E6 B98D 1344 3F4A}

This key needs to be added to the package management key ring of the
system (e.g. {\em apt} on Debian or {\em yum} on RedHat systems).

\subsubsection{\label{sec:installDeb}Installing on Debian/Ubuntu}
There exists a Debian/Ubuntu repository at {\ttfamily
  jwall.org}\footnote{The site \url{http://www.jwall.org/streams/} is
  the base web-site of the \streams framework.} which provides access
to the latest release versions of the \streams library.

To access this repository from within your Debian system, you'll need
create a new file {\ttfamily /etc/apt/sources.list.d/jwall.list} with
the following content:

\sample{ deb http://download.jwall.org/debian/ jwall main } 

The repositories and all packages within the repository are
cryptographically signed with a GPG key. Please see Section
\ref{sec:gpgKey} above for details on how to verify the correctness of
this key.

This key needs to be added to the APT key ring of the Debian/Ubuntu
system by running the following commands (the {\ttfamily \#} denotes
the shell prompt):

\sample{\# sudo wget http://download.jwall.org/debian/software.gpg\newline
\# sudo apt-key add software.gpg 
}

After the key and the repository have been added to the APT package
management, all that is left is to update the package list and install
the \streams environment with the following commands:

\sample{\# sudo apt-get update\newline
\# sudo apt-get install streams
}

The first command will update the package lists, the second will install
the lastest version of the {\ttfamily streams} package. After installation,
the system should be equipped with a new {\ttfamily stream.run} command
to run XML stream processes:

\sample{\# stream.run my-process.xml}

\subsubsection{\label{sec:installRPM}Installing on RedHat/CentOS/Fedora}
There exists a YUM repository at the {\ttfamily jwall.org} site, which
provides access to the latest release versions of the \streams framework
for RedHat based systems.

To access this repository from within your CentOS/RedHat system,
you'll need to create a file {\ttfamily /etc/yum.repos.d/jwall.repo}
with the following contents:

\sample{[jwall]\newline
name=CentOS-jwall - jwall.org packages for noarch\newline
baseurl=http://download.jwall.org/yum/jwall\newline
enabled=1\newline
gpgcheck=1\newline
protect=1}

The RPM packages are signed with a GPG key, please see Section
\ref{sec:gpgKey} for information how to validate this key.

To import the GPG key into your system's key ring, run the
following command as super user:

\sample{\# rpm --import http://download.jwall.org/software.gpg}

After the key has been imported your system is ready to install
the \streams package using the system's package manager, e.g.
by running

\sample{\# yum install streams}

This will download the required packages and set up the system
to provide the {\ttfamily stream.run} command to execute XML
stream processes.