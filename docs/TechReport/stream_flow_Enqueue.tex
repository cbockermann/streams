\Processor{Enqueue}

This processor will enqueue data items into specified queues. To
ensure mutual access to the data, the items are copied and copies are
sent to the queues. This may lead to a multiplication of data.

The processor is a conditioned processor, i.e.~it supports the use of
condition expressions. As an example, the XML snippet in Figure
\ref{fig:ex:Enqueue} will enqueue all events with a {\ttfamily color}
value equal to {\ttfamily blue} into the queue {\ttfamily blue-items}.

\begin{figure}[h!]
        \centering
        \begin{lstlisting}{lang=xml}
           <process ...>
               <Enqueue queues="blue-items" condition="%{data.color} == blue" />
           </process>
        \end{lstlisting}
        \caption{\label{fig:ex:Enqueue}The {\ttfamily Enqueue} processor combined with a condition.}
\end{figure}


\begin{table}[h]
\begin{center}{\footnotesize
{\renewcommand{\arraystretch}{1.4}
\textsf{
\begin{tabular}{|c|c|p{9cm}|c|} \hline
\textbf{Parameter} & \textbf{Type} & \textbf{Description} & \textbf{Required} \\ \hline  
{\ttfamily queues} & ServiceRef[] & A list of names that reference the target queues. & true\\ \hline
{\ttfamily condition} & Condition & A condition that is required to evaluate to {\em true} for this processor to be executed. If no condition is specified, then the processor is executed for every data item. & false\\ \hline
\end{tabular}
 } 
 } 
 } 
\caption{Parameters of class {\ttfamily stream.data.Enqueue}.}
\end{center}
\end{table}
\clearpage