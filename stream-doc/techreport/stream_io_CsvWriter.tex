\paragraph{CsvWriter}

This processor appends all processed data items to a file in CSV format.
The processor either adds all keys of the items or only a set of
previous defined keys.

As first line, the writer emits a header line with a comma separated
list of column names. This line is prepended with a \texttt{\#}
character.

The processor supports the creation of files with varying numbers of
keys/attributes. If an item is processed with a different (larger)
number of keys and the set of keys has not been defined in the
\texttt{keys} parameter, a new header will be inserted into the file,
signaling the header for the next items to be written.

By default the writer uses \texttt{,} as separator, which can be changed
by the \texttt{separator} parameter.

The following example shows a \emph{CsvWriter} writing to
\texttt{/tmp/test.csv} using the \texttt{;} as separator. Only keys
\texttt{@id} and \texttt{name} will be written:

\begin{verbatim}
  <CsvWriter url="file:/tmp/test.csv" keys="@id,name"
             separator=";" />
\end{verbatim}


\begin{figure}[h]
\begin{center}
{\renewcommand{\arraystretch}{1.2}
\textsf{
\begin{tabular}{|c|c|p{9cm}|c|} \hline
\textbf{Parameter} & \textbf{Type} & \textbf{Description} & \textbf{Required} \\ \hline  
keys & String[] & The attributes to write out, leave blank to write out all attributes. & false\\ \hline
url & String & The url to write to. & true\\ \hline
separator & String & The separator to separate columns, usually ',' & false\\ \hline
attributeFilter & String &  & ? \\ \hline
condition & String & The condition parameter allows to specify a boolean expression that is matched against each item. The processor only processes items matching that expression. & false\\ \hline
\end{tabular}
 } 
 } 
\caption{Parameters of processor {\ttfamily CsvWriter}}
\end{center}
\end{figure}
