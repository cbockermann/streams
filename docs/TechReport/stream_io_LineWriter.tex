\paragraph{LineWriter}

This processor simply writes out items to a file in text format. The
format of the file is by default a single line for each item. Any
occurrences of new lines in the values of each item are escaped by
backslash escaping.

The following example will create a single line for each item starting
with some constant string, followed by the value of the items
\texttt{@id} attribute, a constant string \texttt{-\textgreater{}} and
the items \texttt{name} attribute:

\begin{verbatim}
  <LineWriter format="UserId: %{data.@id} -> %{data.name}"
              file="/tmp/example.out.txt"/>
\end{verbatim}


\begin{figure}[h]
\begin{center}
{\renewcommand{\arraystretch}{1.2}
\textsf{
\begin{tabular}{|c|c|p{9cm}|c|} \hline
\textbf{Parameter} & \textbf{Type} & \textbf{Description} & \textbf{Required} \\ \hline  
file & File & Name of the file to write to. & true\\ \hline
append & boolean & Denotes whether to append to existing files or create a new file at container startup. & false\\ \hline
format & String & The format string, containing macros that are expanded for each item & true\\ \hline
escapeNewlines & boolean & Whether to escape newlines contained in the attributes or not. & false\\ \hline
\end{tabular}
 } 
 } 
\caption{Parameters of processor {\ttfamily LineWriter}}
\end{center}
\end{figure}
