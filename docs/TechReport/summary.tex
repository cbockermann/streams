%%
%%
%%
\section{\label{sec:summary}Summary and Future Work}
In this report we introduced the \streams framework, which provides
means for abstracting the definition of data flow graphs for data
stream processing. The level of abstraction provided by the \streams
framework enables a rapid prototyping of compute graphs for process
design as well as providing a simple programming API to include
custom functionality into the designed processes.

The use of XML for process/graph definitions supports a simple
exchange of designed processes between users and lifts the level of
detail for data analysists to hide implementation details where they
may be distracting from the process design task.

The \streams framework also provides a reference implementation for
running compute graphs on a single Java virtual machine as well as a
compiler for mapping graphs to topologies that execute on the {\em
  Storm} stream engine.

The integration of the {\em MOA} library adds various online learning
schemes to the \streams framework. This shows the applicability of the
proposed abstraction layer for the field of online learning. In addition
the \streams library proved to be useful in application use-cases like
pre-processing of the FACT telescope data or the coffee machine video
processing.

\bigskip

Ongoing work currently focuses on a more extensive integration of
additional algorithms provided by {\em MOA} (e.g. clustering). The
adaption of the \streams runtime for the Android platform has revealed
a prototype for running XML process definitions on mobile devices. This
is another direction that will be integrated into the next release of
the \streams framework.

For a more convenient design of \streams process definitions, we will
investigate different XML or process editors that can assist users in
rapid prototyping of data stream processes.