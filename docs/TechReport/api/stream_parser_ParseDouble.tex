\subsubsection{ParseDouble}

This simple processor parses all specified keys into double values. If a
key cannot be parsed to a double it will be replaced by
\emph{Double.NaN}.

The processor will be applied for all keys of an item unless the
\texttt{keys} parameter is used to specify the keys/attributes that
should be transformed into double values.

The following example shows a \emph{ParseDouble} processor that converts
the attributes \texttt{x1} and \texttt{x2} into double values:

\begin{verbatim}
  <ParseDouble keys="x1,x2" />
\end{verbatim}
\subparagraph{Different Default Value}

The \texttt{default} parameter allows for specifying a different value
than the default \emph{Double.NaN} value. The following example converts
all values to their double representation and defaults to 0.0 if parsing
as double fails:

\begin{verbatim}
  <ParseDouble default="0.0" />
\end{verbatim}


\begin{figure}[h]
\begin{center}
{\renewcommand{\arraystretch}{1.2}
\textsf{
\begin{tabular}{|c|c|p{9cm}|c|} \hline
\textbf{Parameter} & \textbf{Type} & \textbf{Description} & \textbf{Required} \\ \hline  
default & Double & The default value to set if parsing fails & false\\ \hline
keys & String[] & The keys/attributes to perform parsing on & true\\ \hline
\end{tabular}
 } 
 } 
\caption{Parameters of processor {\ttfamily ParseDouble}}
\end{center}
\end{figure}
