\subsubsection{AsJSON}

This processor will serialize the current item into a JSON string and
add this string to the item under the key specified in the \texttt{key}
parameter. If no \texttt{key} parameter is provided, the JSON string
will be added with key \texttt{@json}.

This can for example be useful to store the JSON string of an item in a
file or database table in connection with the \emph{SQLWriter}
processor.

The following example shows the serialization into JSON and storage of
the item in a database using the \emph{AsJSON} and \emph{SQLWriter}
processor. In Addition, the \emph{CreateID} processor is used to add an
ID to the item before storing it into the database table
\texttt{data\_items}:

\begin{verbatim}
  <AsJSON key="@json" />
  <CreateID key="@id" />

  <SQLWriter url="jdbc:mysql://localhost:3306/test_db"
             username="TEST" password="TEST" table="data_items"
             keys="@id,@json" />
\end{verbatim}


\begin{figure}[h]
\begin{center}
{\renewcommand{\arraystretch}{1.2}
\textsf{
\begin{tabular}{|c|c|p{9cm}|c|} \hline
\textbf{Parameter} & \textbf{Type} & \textbf{Description} & \textbf{Required} \\ \hline  
key & String & The attribute into which the JSON string of this item should be stored. Default is '@json'. & false\\ \hline
\end{tabular}
 } 
 } 
\caption{Parameters of processor {\ttfamily AsJSON}}
\end{center}
\end{figure}
