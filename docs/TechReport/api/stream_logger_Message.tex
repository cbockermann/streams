\subsubsection{Message}

This processor prints a simple message to standard output. The message
is printed for each processed item and may contain macros that will be
evaluated for each item.

The \emph{Message} processor is a conditioned processor, which allows
for specifying conditions for its execution.

The following example defines a processor that will output a messags
with the items \texttt{@id} attribute for each item that has an
\texttt{alert} value higher than 10:

\begin{verbatim}
 <Message message="Alert for item %{data.@id}!"
          condition="%{data.alert} @gt 10" />
\end{verbatim}


\begin{figure}[h]
\begin{center}
{\renewcommand{\arraystretch}{1.2}
\textsf{
\begin{tabular}{|c|c|p{9cm}|c|} \hline
\textbf{Parameter} & \textbf{Type} & \textbf{Description} & \textbf{Required} \\ \hline  
message & String &  & ? \\ \hline
condition & String & The condition parameter allows to specify a boolean expression that is matched against each item. The processor only processes items matching that expression. & false\\ \hline
\end{tabular}
 } 
 } 
\caption{Parameters of processor {\ttfamily Message}}
\end{center}
\end{figure}
