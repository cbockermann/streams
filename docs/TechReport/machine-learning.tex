\clearpage
\section{\label{sec:machineLearning}Machine Learning with Continuous Data}
As the large volumes of data are merely handable with automatic
processing of that data, they are far away from being inspected
manually. On the other hand gaining insight from that data is the key
problem in various application domains.

Machine learning and data mining has put forth a plethora of
techniques and algorithms for pattern recognition, learning of
prediction model or clustering that all aim at exactly that key
problem: knowledge discover from data.

For the setting of continuous data, various algorithms have been
proposed which solve basic tasks inherent to the knowledge discovery
process as well as complex methods that allow for training classifiers
or finding clusters on steady streams of data. In this section we will
give an overview of how machine learning algorithms are embedded into
the \streams framework using a simple Naive Bayes classifier as example.
% that have been
%implemented within the \streams framework as well as the integration
%of an existing library, namely the MOA \cite{moa}, which provides a
%rich set of state of the art online learning algorithms. 

Following that, we outline the existing online learning implementations
provided by the {\em streams-analysis} package in Section \ref{sec:streamsAnalysis}.
A large set of online learning methods is already provided by the {\em MOA}
library, which is directly integrated into the {\em streams-analysis} package.
We give details on the integration of {\em MOA} in Section \ref{sec:moa}.

%After that overview in Section \ref{sec:onlineLearning}, we discuss
%the problem of the {\em online application} of machine learning models
%which have either been trained online of offline, and then are used to
%make real-life predictions on streaming data. This will be covered in
%Section \ref{sec:onlineApplication}.

\subsubsection*{Notation used}
Within this section, we will denote each data item $d_i$ obtained from
a stream as a tuple $d_i = (d_{i,1},\ldots,d_{i,k}) \subset
M_1\times\ldots\times M_k$, where each $M_i$ might refer to some
domain, e.g. $M_i \subseteq \mathbb{R}$ or an arbitrary set. Further
we will refer to the index $i$ as the index of the item $d_i$ in a
data stream $D$, i.e.
\begin{displaymath}
  D = \langle d_0,d_1,\ldots \rangle.
\end{displaymath}


%%
%%
%%
\subsection{\label{sec:onlineLearning}Online Learning from Data Streams}
The general learning tasks in online learning do not differ from the
traditional objectives. Supervised learning such as classification or
regression tasks need...

Learning from unbounded and continuous data poses tight challenges to
the designer of machine learning algorithms. Even simple basic
building blocks like the computation of a median or minima/maxima
values that might be required in a learning algorithms tend to become
difficult.


\subsubsection*{Approximating Distributions}
As a simple example, the {\em NaiveBayes}\cite{NB} classifier often
offers a adequate prediction performance in a lot of application
domains. Starting with the independence assumption of attributes, it
maximizes computes the class probabilities given the observed
attributes of a set of training instances. The base rule of bayes is
given in equation (\ref{eqn:naiveBayes}):
\begin{eqnarray}
  P(c | f_1,\ldots,f_n ) = \frac{P(c)\cdot P(f_1,\ldots,f_n|c)}{P(f_1,\ldots,f_n)}.\label{eqn:naiveBayes}
\end{eqnarray}

Assuming a fixed set $C = \{c_1,\ldots,c_l\}$ of observable classes, we
can easily approximate $P(c)$ by counting the occurences for each
$c_i$ in the observed stream. For pure numerical attributes $f_i$,
generally a gaussian normal distribution is assumed, such that the
factors $P(f_1,\ldots,f_n|c)$ and $P(f_1,\ldots,f_n)$ can be derived
by estimating the mean and average for each attribute $f_i$.

%The setting gets a bit more complicated if the $f_i$ are nominal
%values, such as variable strings from an unbounded domain such as
%URLs, i.e. $M_i \cong \mathbb{N}$. In this case we cannot simply
%derive a probability for each instance of an attribute as this would
%require counting an unbounded set of strings, which clearly violates
%the stream processing contraints mentioned in Section
%\ref{sec:streamSetting}.

%%
%% How are classifiers embedded into the streams framework?
%% How can they be used?
%%
\subsubsection{Embedding Classifiers in \streams}


%%
%% Which classifiers/clusterers/etc. are available?
%%
\subsubsection{Available Online Learning Algorithms}

\subsubsection*{Online Statistics (Counting, Quantiles)}

\subsubsection*{Online Classifier}

\include{machine-learning-streams}

\subsection{\label{sec:moa}Integrating MOA}

MOA is a software package for online learning tasks. It provides a
large set of clustering and classifier implementations suited for
online learning. Its main intend is to serve as an environment for
evaluating online algorithms.

The \streams framework provides the {\ttfamily stream-analysis}
artifact, which includes MOA and allows for integrating MOA
classifiers directly into standard stream processes.

The following example XML snippet shows the use of the Naive Bayes
implementation of MOA within a \streams container. The example defines
a standard test-then-train process.


\begin{figure}[h!]
  \centering
  \begin{lstlisting}[language=XML]
      <container>
           <stream id="stream" class="stream.io.CsvStream"
                   url="classpath:/multi-golf.csv.gz" limit="100"/>

           <process input="stream">
                <RenameKey from="play" to="@label" />
        
                <stream.learner.Prediction learner="NB" />

                <stream.learner.evaluation.PredictionError />

                <moa.classifiers.bayes.NaiveBayes id="NB"/>

                <stream.statistics.Sum keys="@error:NB" />
           </process>
      </container>
  \end{lstlisting}
  \caption{\label{fig:testThenTraing}}
\end{figure}

\subsubsection{The {\ttfamily moa} packages}
The {\ttfamily stream-analysis} module of the \streams library uses a
simple wrapper approach to integrate the MOA classes into the streams
framework. All implementations of MOA are mapped to their default Java
package, i.e.


\begin{figure}[h!]
  \centering
  \begin{lstlisting}[language=XML]
   ...
     <process input="..">

         <moa.classifiers.bayes.NaiveBayes />

     </process>
   ...
  \end{lstlisting}
  \caption{\label{fig:moaClassifierXML}}
\end{figure}

The options used in MOA are directly mapped to XML element attributes.


%%
%% What is online model application? When is it required?
%% What methods are provided? RapidMiner-Models?
%%
\subsection{\label{sec:onlineApplication}Model Application on Data Streams}

As previously outlined in Section \ref{sec:onlineLearning}, each
classifier provides a service that can be used to access or use its
model. Such services can for example be a {\em PredictionService},
which provides a prediction function for a data item. The former
section mainly focused on the online training of such classifiers,
whereas in this part, we deal with the application of such
classifier/learning output to online data streams.

There are two key aspects we want to discuss here, namely the use of
classifier models for the evaluation of online learning algorithms in
Section \ref{sec:evalOnline} and the general benefit of applying
models that may even have been trained offline in the setting of
continuous data streams in Section \ref{sec:applyOnline}.


%%
%% Raise some questions on "why" and "when" online model application
%% is required, touch "concept drift" problems,...
%%
%% keywords: monitoring, static models, expert knowledge?
%%

%%
%% A simple but important use of online model application is the 
%% evaluation of classifiers on online data streams.
%%
%% This section should give an example walk-through for test-then-train
%% and demonstrate an example container that evaluates a classifier (MOA?)
%% on a data stream.
%%
\subsubsection{\label{sec:evalOnline}Evaluating Classifiers on a Stream}




%%
%% Section about the need to create a model on offline data and
%% using that model on dynamic continuous data streams. 
%% Outline:
%%   - high-level (short) introduction to a use-case
%%   - outlining the offline training of models
%%   - incorporation of the models into the streams library
%%   - maybe a more detailed example to sum up (plus container definiton in the appendix?)
%%
\subsubsection{\label{sec:applyOnline}Learning Offline, Predicting Online}


