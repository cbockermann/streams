\Processor{Skip}

This processor will simply skip all events matching a given condition.
If no condition is specified, the processor will skip all events.

The condition must be a bool expression created from numerical operators
like \texttt{@eq}, \texttt{@gt}, \texttt{@ge}, \texttt{@lt} or
\texttt{@le}. In addition to those numerical tests the \texttt{@rx}
operator followed by a regular expression can be used.

The general syntax is

\begin{verbatim}
   variable  operator  argument
\end{verbatim}
For example, the following expression will check the value of attribute
\texttt{x1} against the 0.5 threshold:

\begin{verbatim}
   %{data.x1} @gt 0.5
\end{verbatim}


\begin{table}[h]
\begin{center}{\footnotesize
{\renewcommand{\arraystretch}{1.4}
\textsf{
\begin{tabular}{|c|c|p{9cm}|c|} \hline
\textbf{Parameter} & \textbf{Type} & \textbf{Description} & \textbf{Required} \\ \hline  
{\ttfamily condition } & String & The condition parameter allows to specify a boolean expression that is matched against each item. The processor only processes items matching that expression. & false\\ \hline
\end{tabular}
 } 
 } 
 } 
\caption{Parameters of class {\ttfamily stream.flow.Skip}.}
\end{center}
\end{table}
