\Processor{Timestamp}

This processor parses the date time from an attribute using a specified
format string and stores the parsed time as a long value into the
\texttt{@timestamp} key by default.

The processor requires at least a \texttt{format} and a \texttt{from}
parameter. The \texttt{format} specifies a date format to parse the time
from. The \texttt{from} parameter determines the key attribute from
which the date is to be parsed.

The following example shows a timestamp parser that parses the
\texttt{DATE} key using the format \texttt{yyyy-MM-dd-hh:mm:ss}. The
resulting timestamp (milliseconds UNIX time) is stored under key
\texttt{@time}:

\begin{verbatim}
  <Timestamp key="@time" format="yyyy-MM-dd-hh:mm:ss"
             from="DATE" />
\end{verbatim}


\begin{table}[h]
\begin{center}{\footnotesize
{\renewcommand{\arraystretch}{1.4}
\textsf{
\begin{tabular}{|c|c|p{9cm}|c|} \hline
\textbf{Parameter} & \textbf{Type} & \textbf{Description} & \textbf{Required} \\ \hline  
{\ttfamily key } & String &  & false\\ \hline
{\ttfamily format } & String & The date format string used for parsing. & true\\ \hline
{\ttfamily from } & String & The key/attribute from which the timestamp should be parsed. & true\\ \hline
{\ttfamily timezone } & String & The timezone that the processed data is assumed to refer to. & false\\ \hline
\end{tabular}
 } 
 } 
 } 
\caption{Parameters of class {\ttfamily stream.parser.ParseTimestamp}.}
\end{center}
\end{table}
