\subsubsection{NumericalBinning}

This processor transforms a numeric attribute into a discrete nominal
attribute by creating bins and mapping all values onto a bin. This can
be regarded as a simple interval-based discretization.

The \emph{NumericalBinning} processor requires a \texttt{minimum},
\texttt{maximum} and \texttt{bins} parameter to be set. It will then
compute equi-distant interval from these values and will replace each
value by a string denoting the interval/bin it was mapped to.

By default, the \emph{NumericalBinning} will discretize all numeric
attributes, i.e.~all keys that refer to an Integer, Long, Float or
Double value. If the \texttt{keys} parameter is provided, only the
attributes listed in \texttt{keys} are discretized.

The following example shows the numerical binning for attribute
\texttt{x1}:

\begin{verbatim}
    &lt;NumericalBinning keys="x1" minimum="0.0" maximum="10.0" bins="10" /&gt;
\end{verbatim}


\begin{figure}[h]
\begin{center}
{\renewcommand{\arraystretch}{1.2}
\textsf{
\begin{tabular}{|c|c|p{9cm}|c|} \hline
\textbf{Parameter} & \textbf{Type} & \textbf{Description} & \textbf{Required} \\ \hline  
key & String &  & ? \\ \hline
minimum & Double &  & ? \\ \hline
maximum & Double &  & ? \\ \hline
bins & Integer &  & ? \\ \hline
keys & String[] &  & ? \\ \hline
\end{tabular}
 } 
 } 
\caption{Parameters of processor {\ttfamily NumericalBinning}}
\end{center}
\end{figure}
