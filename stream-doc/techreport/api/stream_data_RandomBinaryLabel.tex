\subsubsection{RandomBinaryLabel}

This operator simply adds a random label (either -1.0 or +1.0) to
processed data items. The labels are distributed normally.

The \texttt{key} parameter allows for specifying the label attribute
name, which by default is considered to be \texttt{@label}:

\begin{verbatim}
 <RandomBinaryLabel key="@label" />
\end{verbatim}


\begin{figure}[h]
\begin{center}
{\renewcommand{\arraystretch}{1.2}
\textsf{
\begin{tabular}{|c|c|p{9cm}|c|} \hline
\textbf{Parameter} & \textbf{Type} & \textbf{Description} & \textbf{Required} \\ \hline  
seed & Long &  & ? \\ \hline
key & String &  & ? \\ \hline
\end{tabular}
 } 
 } 
\caption{Parameters of processor {\ttfamily RandomBinaryLabel}}
\end{center}
\end{figure}
