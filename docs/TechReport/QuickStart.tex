%\documentclass[12pt,parskip=half]{scrartcl}
\documentclass{TechReport}

% Select Sans serif font "Computer Modern Bright"
%\usepackage[T1]{fontenc}
%\usepackage{cmbright}

% ***TODO*** Select the right encoding for your document
%\usepackage[utf8]{inputenc}
%\usepackage[latin1]{inputenc}

%\usepackage{graphicx}

% Bibtex citation package
%\usepackage{cite}

%\usepackage{url}

% No page numbers here; these will be added for final Tech Report with
% all contributions
%\renewcommand*{\titlepagestyle}{empty}
%\pagestyle{empty}

%\title{LaTeX-Template for a Technical Report}
%\author{Christian Bockermann}
\date{01/2012}  % Serial Number/Year
\sfbproject{C3} % Your project identifier (two digit version)

\usepackage{amssymb}


% ***TODO*** Choose correct title, author, department and email address
\title{The \textsf{streams} Library for Processing Streaming Data}
\author{Christian Bockermann\\
  Lehrstuhl f\"ur k\"unstliche Intelligenz\\
  Technische Universit\"at Dortmund\\*[1ex]
  christian.bockermann@udo.edu}
% Please keep date empty
\date{}





\usepackage{listings}
\usepackage{hyperref}
\hypersetup{
  linktoc=all
}

\usepackage{tikz}
\usetikzlibrary{shapes} 
\usetikzlibrary{decorations,arrows} 
\usetikzlibrary{decorations.pathmorphing} 
\usepgflibrary{decorations.pathreplacing} 

\usepackage{algorithm}
%\usepackage{algorithmicx}
\usepackage{algpseudocode}

\usepackage{color}
\definecolor{rapidi}{RGB}{240,176,0}
\definecolor{rapidiText}{RGB}{102,51,0}
\definecolor{darkGreen}{RGB}{67,101,0}

\DeclareRobustCommand{\pointer}[1]{\tikz\node[draw,circle,draw=darkGreen,thick,fill=darkGreen!70,inner sep=1pt] at (0,-0.1) {\scriptsize\color{white}\textsf{#1}};} 

\lstset{language=XML,
%  otherkeywords={stream,container,process,service},
  basewidth={0.5em,0.45em},
  fontadjust=false,
  basicstyle=\footnotesize\ttfamily,
  keywordstyle=\color{blue},          % keyword style
  commentstyle=\ttfamily\color{black!60},       % comment style
  stringstyle=\color{darkGreen}
}


\newcommand{\todo}[1]{\marginpar{{\bf TODO:}\\#1}}
%\newcommand{\streams}{{\footnotesize\textsf{streams}\ }}
\newcommand{\streams}{{\em streams}\ }
\newcommand{\rapidminer}{{\footnotesize\textsf{RapidMiner}\ }}
%\newcommand{\plugin}{{\footnotesize\textsf{Streams Plugin}}}
\newcommand{\plugin}{{\em Streams Plugin}}

\newcommand{\defitem}[1]{\item[{\footnotesize\textbf{\textsf{#1}}}]}
\newcommand{\bigO}{\cO}
\newcommand{\barw}{\bar{w}}
\newcommand{\barbarw}{\bar w_M^\circ}
\newcommand{\vx}{x}


\begin{document}

\makesfbtitlepage

\newpage
\tableofcontents
\newpage

\begin{abstract}
{\normalsize
  In this report, we present the \streams library, a generic
  Java-based library for designing data stream processes. The \streams
  library defines a simple abstraction layer for data processing and
  provides a small set of online algorithms for counting and
  classification. Moreover it integrates existing libraries such as
  MOA. Processes are defined in XML files which follow the
  semantics and ideas of well established tools like Maven or the
  Spring Framework.

  The \streams library can be easily embedded into existing software,
  used as a standalone tool or be integrated into the RapidMiner tool
  using the \plugin.}
\end{abstract}

%\begin{document}
%\maketitle

%%%
%% Introduction
%%   - Motivation/Use Cases
%%   - constraints of data stream processing
%%   - our contributions (list)
%%
\section{Introduction}
More and more applications rely on dynamic data that is produced in
realtime and at a high volume. Scientific experiments, network
traffic, sensor networks in manifacturing processes or message
services are examples of such applications. Often the data in these
applications is outdated quickly and reactions need to be applied in
near to realtime. An example is given by Google's news search, which
uses a dynamic index for searching even more recent news articles. In
other scenarios an on-time analysis might save resources as irrelevant
data can quickly be detected and discarded. Analysis in such dynamic
data settings is different to the traditional batch setting that
RapidMiner has initially been designed for.

Continuous data poses several challenges for data analysts: The data
are often produced at large volume and require continuous processing
to provide up-to-date prediction models or summaries. Such models or
statistics need to be accessible at anytime. For preprocessing that
data only limited resources with regard to memory, CPU and I/O is
available. Recent advances such as Google's Map/Reduce paradigm
address these by large scale parallelization of batch processes
\cite{googleMapReduce,radoop}. While this scales well with the large
amounts of data at hand, it does not tackle the problem of processing
data {\em continuously}.

%The massive amounts of data continuously produced by scientific
%experiments or in large scale applications demand for alternative
%approaches to traditional random-access and in-memory data analysis.
%As an example, the FACT project\cite{fact} runs a telescope recording
%up to one terabyte of raw data per night. As analysis of this data
%just started, it is not yet clear which parts of the data might be
%useful and which might be subject to being discarded.

To catch up with the reqirements of large scale and continuous data,
online algorithms have recently received a lot of attention. Various
algorithms have been proposed for online quantile computation
\cite{Greenwald/Khanna/2001a,Arasu/Manku/2004a}, frequent itemset
mining
\cite{Charikar02findingfrequent,goethals2007,Cheng06maintainingfrequent},
clustering \cite{sohler2010,Aggarwal:2003} or classification
\cite{Domingos/Hulten/2000a}.


\subsection{Our Contributions}
In this work we introduce the \streams library, a small software
framework that focuses on online processing of data and its adaption
into RapidMiner as the \plugin. The \streams framework provides a
thin abstraction layer to facilitate online data processing whereas
the \plugin\ uses a generic wrapper approach\footnote{RapidMiner
  operators are automatically generated using the \textsf{RapidMiner
    Beans} library, which allows for the implementation of operators
  by following the JavaBeans convention and using simple Java
  annotations.} to build a streaming facade within RapidMiner.

The proposed library supports
\begin{enumerate}
\item Modelling of continuous stream processes within RapidMiner,
  following the {\em single-pass} paradigm,
\item Anytime access to services that are provided by the continuous
  processing and the online algorithms deployed in the process setup,
  and
\item Processing of large data sets using limited memory resources.
\end{enumerate}

\subsection{Paper Outline}
The outline of this work is as follows: In Section
\ref{sec:relatedWork} we review the problem setting and give an
overview of related work and existing frameworks.
%% introduce some of the main objectives in data
%stream mining and provide some example use cases. 
Based on this we derive some basic building blocks for a modeling data
stream processes (Section \ref{sec:abstraction}). In Section
\ref{sec:streamsLibrary} we present the \streams API which provides
implementations to these building blocks, and present the \plugin\
that integrates these into RapidMiner in Section
\ref{sec:plugin}. Finally we summarize the ideas behind the \streams
library and give an outlook on future work.
%  for  main ideas and
%paradigms inherent to the \streams library.
% . Based on this we describe the data model and illustrate the
% architecture of the
%Finally we provide some use-cases for the RapidMiner \plugin\ in
%Section \ref{sec:usecases} and give an outlook on future work.


\section{Introduction\label{sec:intro}}
Todays applications produce a plethora of data on a continuous basis:
In the field of astrophysics, telescopes produce raw data that quickly
exceeds hundreds of gigabytes in a single day \cite{factTelescope}. In
the area of social networks, Twitter announced recently to have
observed a peak of 150,000 messages per minute during the TV duell
between president Obama and his challenger Romney \cite{}. In
computing farms and large scale networks there is a large pile of log
data about system state, system activities or working load gathered
and stored that is far from being analyzed by any human, but may
embody valuable information about the system
\cite{googleHDprediction}.

The data we are considering here, is non-stationary, in the sense that
new data is continously produced. This is fundamentally different to
the traditional batch-data processing paradigm that has been
dominating the data-analysis and data-mining world for the past
decades. That paradigm shift has triggered the evolving area of stream
or online analysis, which investigates algorithms for analysis of
continuous data.

Principles in DSM

In order to support the design of processes that allow for online
analysis we propose the \streams library. It is a Java based
abstraction layer for defining stream processes by means of simple XML
definitions. It generally follows the {\em pipes-and-filters} pattern
for data flow processing and extends this by an orthogonal control
flow view to support the notion of anytime services within the
process design.




%\section{\label{sec:intro}Introduction}
In todays applications, data is continuously produced in various spots
ranging from network traffic, log file data, monitoring of
manufacturing processes or scientific experiments. The applications
typically emit data in non-terminating data streams at a high rate,
which poses tight challenges on the analysis of such streams.

Several projects within the Collaborative Research Center SFB-876 deal
with data of large volume. An example is given by the FACT telescope
that is associated to project C3. This telscope observes cosmic
showers by tracking light that is produced by these showers in the
atmosphere with a camera.  These showers last about 20 nanoseconds and
are recorded with a camera of 1440 pixels at a sampling rate of 2
GHz. As about 60 of these showers are currently recorded each second,
a 5-minute recording interval quickly produces several gigabytes of
raw data.

Other high-volume data is produced in monitoring system behavior, as
performed in project A1. Here, operating systems are monitored by
recording fine grained logs of system calls to catch typical usage of
the system and optimize its resource utilization (e.g. for energy
saving). System calls occur at a high rate and recording produces a
plethora of log entries.

The project B3 focuses on monitoring (distributed) sensors in an
automated manufacturing process. These sensors emit detailed
information of the oven heat or milling pressure of steel production
and are recorded at fine grained time intervals. Analysis of this data
focuses on supervision and optimization of the production process.

\subsection{From Batches to Streams}
Traditional data analysis methods focus on processing fixed size
batches of data and often require the data (or large portions of it)
to be available in main memory. This renders most approaches useless
for continuously analyzing data that arrives in steady streams. Even
procedures like preprocessing or feature extraction can quickly become
challenging for continuous data, especially when only limited
resources with respect to memory or computing power are available.

To catch up with the reqirements of large scale and continuous data,
online algorithms have recently received a lot of attention. The focus
of these algorithms is to provide approximate results while limiting
the memory and time resources required for computation. The
constraints for the data stream setting are generally defined by
allowing only a single pass over the data, and focusing on
approximation schemes to deal with the inbalance of data volume to
computing resources. In addition, models computed on streaming data
are expected to be queriable at any time.

Various algorithms have been proposed dedicated to computational
problems on data streams. Examples include online quantile computation
\cite{Greenwald/Khanna/2001a,Arasu/Manku/2004a}, distinct counting of
elements, frequent itemset mining
\cite{Charikar02findingfrequent,goethals2007,Cheng06maintainingfrequent},
clustering \cite{sohler2010,Aggarwal:2003} or training of classifiers
on a stream \cite{Domingos/Hulten/2000a}.


\subsection{Designing Stream Processes}
In this work we introduce the \streams library, a small software
framework that provides an abstract modelling of stream processes. The
objective of this framework is to establish a layer of abstraction
that allows for defining stream processes at a high level, while
providing the glue to connect various existing libraries such as MOA
\cite{moa}, WEKA \cite{weka} or the RapidMiner tool.

The set of existing online algorithms provides a valuable collection
of algorithms, ideas and techniques to build upon. Based on these core
elements we seek to design a process environment for implementing
stream processes by combining implementations of existing online
algorithms, online feature extraction methods and other preprocessing
elements.
%or implement and evaluate custom online algorithms 

Moreover it provides a simple programming API to implement and
integrate custom data processors into the designed stream processes.
The level of abstraction of this programming API is intended to
flawlessly integrate into existing runtime environments like {\em
  Storm} or the RapidMiner platform \cite{rapidminer}.

Our proposed framework supports
\begin{enumerate}
\item Modelling of continuous stream processes, following the {\em
    single-pass} paradigm,
\item Anytime access to services that are provided by the modeled
  processes and the online algorithms deployed in the process setup,
  and
\item Processing of large data sets using limited memory resources
\item A simple environment to implement custom stream processors and
  integrate these into the modelling
\item A collection of online algorithms for counting and classification
\item Incorporation of various existing libraries (e.g. MOA
  \cite{moa}) into the modeled process.
\end{enumerate}

The rest of this report is structured as follows: In Section
\ref{sec:relatedWork} we review the problem setting and give an
overview of related work and existing frameworks.
%% introduce some of the main objectives in data
%stream mining and provide some example use cases. 
Based on this we derive some basic building blocks for a modeling data
stream processes (Section \ref{sec:abstraction}). In Section
\ref{sec:streamsLibrary} we present the \streams API which provides
implementations to these building blocks. Finally we summarize the
ideas behind the \streams library and give an outlook on future work.

%
%%%
%% In this section:
%%
%%   - explain stream setting, problems and objectives/aims
%%   - theoretical frameworks?
%%   - stream processes, continuous systems
%%   - Frameworks
%%        - batch processing: RM, Weka
%%        - mini-batches + parallelization:  MapReduce,Hadoop,Radoop
%%        - stream processing: moa,s4.io,storm,Streams-Plugin
%%
%%   - ggf. �bersichtstabelle mit den Ans�tzen und Einordnen des Streams Plugin?
%%
%% (  - refine the basic problems and task/settings we want to address )
%% (  - give an overview of existing frameworks/solutions )
%% (  - outline the differences of the streams library to this solutions )
%%
\section{\label{sec:relatedWork}Problem and Related Work}

%
%. The {\em data items} $s_i$ are tuples of
%$M^p$ with $p \ge 1$ where $M^p = M_1 \times \ldots \times M_p$ for
%any sets $M_j$. 
%
%The $M_j$ can be any discrete sets of a fixed domain as well as $M_k
%\subseteq \mathbb{R}$. The index $i$ may reflect some time unit or an
%(monotonically increasing) time like dimension, constituting the
%sequence of tuples. For $p=1$ and $M_1 = \mathbb{R}$ this models a
%single value series with index $i$.
%The data processing model of streaming approaches share common
%criteria.
% induced by the continuous nature of the problem. 

Several algorithms to the task mentioned above with respect to these
requirements have been proposed. Regarding the counting of elements
and sets of items, a variety of different approximate count algorithms
based on sketches of the data have been developed in
\cite{Charikar02findingfrequent,goethals2007,Cheng06maintainingfrequent}.
For statistical model, estimators for quantiles have been presented in
\cite{Greenwald/Khanna/2001a,Arasu/Manku/2004a}.



\bigskip

Whereas a wide range of different methods have been provided for
various streaming tasks, this work aims at providing an abstract
framework to integrate these different approaches into a flexible
environment to build a streaming analysis based upon the existing
algorithms.

%%The examples outlined above show typical use cases of high-volume
%%continues data that is either stored in large batches (e.g. 5-minute
%%recording intervals of the FACT telescope) or continuously produced
%%by non-terminating processes.
%%
%%The results is a sequence of 150 images (slices), each containing 1440
%%pixels.  As the shower is not clearly identifiable, a {\em region of
%%  interest} of about 300 slices is recorded, which results in 432000
%%raw data values for a single shower (event). 
%%
%%Currently showers are recorded at a rate of 60 Hz, resulting in 60 of
%%such events being stored each second. With additional information for
%%each shower, a 5 minute recording interval quickly produces several
%%gigabytes of raw data that is to be preprocessed and analyzed.
%%
%With nowadays data volume, the traditional batch processing model
%quickly reaches the resource limitations of single workstations. Even
%applying a previously created prediction model to a large set of
%examples can quickly become impossible if the example set itself does
%not fit into main memory. The only cumbersome solution often is to
%split the data into several files and process each file separately.
%We will refer to this setting as the {\em partial batch processing}.
%This processing typically requires the results of the processed
%batches to be combined, for example by computing an average.
%
%In some cases, the data is not even static, but continuosly produced
%by some data generating process. In the simplest case we might be able
%to write batches of that data into files and fall back to the mini
%batch processing approach. Therefore in this work we are more
%interested in continuously processing that data and provide models or
%services in an {\em anytime} manner, that is the current models or
%statistics can be queried at any time. We will refer to this setting
%as the ({\em continuous}) {\em stream processing}.  When dealing with
%a finite source of data we can consider the {\em stream processing} as
%a special case of {\em partial batch processing} with a batch size of 1.
%
%The data processing model of streaming approaches share common
%criteria.
% induced by the continuous nature of the problem. 
%The framing to operate on streaming data is generally given by the
%following constraints/requirements:
%\begin{itemize}
%  \item[\textsf{C1}] continuously processing {\em single items} or {\em small batches} of data,
%  \item[\textsf{C2}] using only a {\em single pass} over the data,
%  \item[\textsf{C3}] using {\em limited resources} (memory, time),
%  \item[\textsf{C4}] provide {\em anytime services} (models, statistics).
%\end{itemize}
%This contrasts to the RapidMiner batch-processing model, where a set
%of examples is usually processed in its entirety and during a single
%execution of a RapidMiner process.


\subsection*{Existing Frameworks}
%Various frameworks exist that support either of these two processing
%modes.
Parallel batch processing is addressing the setting of fixed data and
is of limited use if data is non-stationary but continuously produced,
for example in monitoring applications (server log files, sensor
networks).  A framework that provides online analysis is the MOA
library \cite{moa}, which is a Java library closely related to the
WEKA data mining framework \cite{weka}. MOA provides a collection of
online learning algorithms with a focus on evaluation and
benchmarking.

Aiming at processing high-volume data streams two environments have
been proposed by Yahoo! and Twitter. Yahoo!'s {\em S4} \cite{s4io}
as well as Twitter's {\em Storm} \cite{storm} framework do provide
online processing and storage on large cluster infrastructures, but
these do not include any online learning.

In contrast to these frameworks, the \streams library focuses on
defining a simple abstraction layer that allows for the definition of
stream processes which can be mapped to different backend
infrastructures (such as {\em S4} or {\em Storm}).

%Providing an execution environment for data stream processing is given
%in s4.io \cite{s4io} and {\em storm} \cite{storm}. These libraries
%... \todo{More details about s4io/storm}.



%%%
%% Structure of this section
%%
%%  3.1 define basics, abstraction of stream processing
%%    - Def. data items, data streams,...
%%    - data flow vs. control flow => Bezug zu Anytime
%%    - Warum keine "Meta-Daten" (bisher)
%%
%%  3.2 Umsetzung/Modellierung in Form der Stream-API
%%    - Abbildung der Konzepte aus 3.1 auf "Java Welt"
%%  
%%  3.3 Example Runtime for RapidPrototyping, Debugging
%%    + Ausblick auf section 4 => streamplugin + runtime für RapidMiner
%%
%%
\section{\label{sec:abstraction}An Abstract Stream Processing Model}
In this section we introduce the basic concepts and ideas that we
model within the \streams framework. This mainly comprises the data
flow (pipes-and-filters \cite{softwarePatterns}), the control flow
(anytime services) and the basic data structures and elements used for
data processing. The objective of the very simple abstraction layer is
to provide a clean and easy-to-use API to implement against. 

The structure of the \streams framework builds upon thress basic parts:
\begin{enumerate}
\item A {\em data representation} which provides a modeling of the
  data that is to be processed by the designed stream processes
\item Elements to model a {\em data flow}
\item A notion of {\em services} which allow for the implementation
  of {\em anytime service} capabilities.
\end{enumerate}
All of these elements are provided as simple facades (interfaces)
which have default implementations. The abstraction layer provided by
these facades is intended to cover most of the use cases with its
default assumptions, whereas any special use cases can generally be
modelled using a combination of different building blocks of the API
(e.g. queues, services) or custom implementations of the facades.

\subsection{\label{sec:data}Data Representation and Streams}
A central aspect of data stream processing is the representation of
data items that contain the values which are to be processed.  We
consider the case of continuous streaming data being modeled as a
sequence of {\em data items}. 

A data item is a set of $(k,v)$ pairs, where each pair reflects
an attribute with a name $k$ and a value $v$. The names are required
to be of type {\ttfamily String} whereas the values can be of any
type that implements Java's {\ttfamily Serializable} interface. The
data item is provided by the {\ttfamily stream.data.Data} interface.

\begin{figure}[h!]
\begin{center}{
\renewcommand{\arraystretch}{1.25}
\begin{tabular}{c|c}\hline
\textsf{\textbf{Key}} & \textsf{\textbf{Value}} \\ \hline \hline
{\ttfamily x1} & {\ttfamily 1.3} \\ \hline
{\ttfamily x2} & {\ttfamily 8.4} \\ \hline
{\ttfamily source} & {\ttfamily "file:/tmp/test.csv"}  \\ \hline
\end{tabular}
}
\end{center}
\caption{\label{tab:dataitem}A data item example with 3 attributes.} 
\end{figure}

Figure \ref{tab:dataitem} shows a sample data item as a table of
(key,value) rows. This representation of data items is provided
by hash tables, which are generally provided in almost every modern
programming language.

The use of such hash tables was chosen to provide a flexible data
structure that allows for encoding a wide range of record type as well
as supporting easy interoperation when combining different programming
languages to implement parts of a streaming process.

\subsubsection*{Streams of Data}
A {\em data stream} in consequence is an entity that provides access
to a (possibly unbounded) sequence of such data items. Again, the
\streams abstraction layer defines data streams as an interface, which
essentially provides a method to obtain the next item of a stream.

\begin{figure}[h1]
  \begin{center}
    \includegraphics[scale=0.5]{graphics/stream-items.png}
  \end{center}
  \caption{\label{fig:datastream}A {\em data stream} as a sequence of {\em data item} objects.}
\end{figure}

The core \streams library contains several implementations for data
streams that reveal data items from numerous formats such as CSV data,
SQL databases, JSON or XML formatted data. A list of the available
data stream implementations is available in the appendix \ref{app:dataStreams}.

In addition, application specific implementations for data streams can
easily be provided by custom Java classes, as is the case in the FACT
telescope data use-case outlined in section \ref{sec:fact}.

%Throughout this work, we will denote a {\em
%  data stream} by $D$ and a family of such streams as $D_l$. A data
%stream $D$ is a sequence
%\begin{displaymath}
%  D = \langle d_0,d_1,\ldots,d_i,\ldots \rangle
%\end{displaymath}
%of data items $d_i$. Let $A = \{A_1,\ldots,A_p\}$ be a set of
%attributes, then a data item $d_i$ is a function mapping attributes to
%values of a domain associated with each attribute. In database
%notation, each $d_i$ is a tuple of $M^p = M_{A_1} \times\ldots\times
%M_{A_p}$ for sets $M_j$. The $M_j$ can be any discrete sets of a fixed
%domain as well as $M_j \subseteq \mathbb{R}$. The values of each data
%item are further denoted by $d_i(k)$, i.e.
%\begin{displaymath}
%  d_i = ( d_i(A_1),\ldots,d_i(A_p) ).
%\end{displaymath}
%
%The index $i$ may reflect some time unit or a (monotonically
%increasing) time-like dimension, constituting the sequence of
%tuples. For $p=1$ and $M_1 = \mathbb{R}$ this models a single value
%series with index $i$.


%
%. The {\em data items} $s_i$ are tuples of
%$M^p$ with $p \ge 1$ where $M^p = M_1 \times \ldots \times M_p$ for
%any sets $M_j$. 
%

\subsection{\label{sec:basics}Processes and Processors}
The {\em data stream}s defined above encapsulate the format and reading
of data items from some source. The \streams framework defines a
{\em process} as the consumer of such a source of items. A process is
connected to a stream and will apply a series of {\em processors} to
each item that it reads from its attached data stream.

Each {\em processor} is a function that is applied to a data item and
will return a (modified or new) data item as a result. The resulting
data item then serves as input to the next processor of the
process. This reflects the pipes-and-filters concept mentioned in the
beginning of this section.

The {\em processors} are the low-level functional units that actually
do the data processing and transform the data items. There exists a
variety of different processors for manipulating data, extracting or
parsing values or computing new attributes that are added to the data
items.  From the prespective of a process designer, the {\em stream}
and {\em process} elements form the basic data flow elements whereas
the processors are those that do the work.

\begin{figure}[h!]
\centering
\includegraphics[scale=0.5]{graphics/inside-process.png}
\caption{\label{fig:process}A process reading from a stream and applying processors.}
\end{figure}

The simple setup in Figure \ref{fig:process} shows the general role of
a process and its processors. In the default implementations of the
\streams library, this forms a {\em pull oriented} data flow pattern
as the process reads from the stream one item at a time and will only
read the next item if all the inner processors have completed. 

Where this pull strategy forms a computing strategy of {\em lazy
  evaluation} as the data items are only read as they are processed,
the \streams library is not limited to a {\em pull oriented} data
flow. We discuss the implementation of {\em active streams} and the
resulting {\em push} oriented data flow in Section \ref{sec:push}.


\subsubsection*{Using multiple Processes}
In the \streams framework, processes are by default the only executing
elements. A process reads from its attached stream and applies all
inner processors to each data item. The process will be running until
no more data items can be read from the stream (i.e. the stream
returns {\ttfamily null}). Multiple streams and processes can be
defined and executing in parallel, making use of multi-core CPUs as
each process is run in a separate thread\footnote{This is the default
  behavior in the reference \streams runtime implementation. If
  \streams processes are executed in other environments, thus behavior
  might be subject to change.}.

For communication between processes, the \streams environment provides
the notion of {\em queues}. Queues can temporarily store a limited
number of data items and can be fed by processors. They do provide
stream functionality as well, which allows queues to be read from by
other processes.

Figure \ref{fig:queues} shows two processes being connected by a
queue. The enlarged processor in the first process is a simple {\em
  Enqueue} processor that pushes a copy of the current data item into
the queue of the second process. The second process constantly reads
from this queue, blocking while the queue is empty.

\begin{figure}[h!]
  \begin{center}
    \includegraphics[scale=0.5]{graphics/process-queues.png}
  \end{center}
  \caption{\label{fig:queues}Two Processes $P_1$ and $P_2$ communicating via queues.}
\end{figure}

\bigskip

These five basic elements ({\em stream}, {\em data item}, {\em
  processor}, {\em process} and {\em queue}) already allow for
modelling a wide range of data stream processes with a sequential and
multi-threaded data flow.
%This contrasts to the current RapidMiner execution model, where each
%operator within a process is executed only once (not counting loops as
%within a cross validation).
%This simple {\em data flow} view serves as the basic data-driven
%exectuion model. 
Apart from the continuous nature of the data stream source, this model
of execution matches the same pipelining idea known from tools like
RapidMiner, where each processor (operator) performs some work on a
complete set of data (example set).

\subsection{Data Flow and Control Flow}
A fundamental requirement of data stream processing is given by the
{\em anytime paradigm}, which allows for querying processors for their
state, prediction model or aggregated statistics at any time. We will
refer to this anytime access as the {\em control flow}.  Within the
\streams framework, these anytime available functions are modeled as {\em
  services}. A service is a set of functions that is usually provided
by processors and which can be invoked at any time. Other processors
may consume/call services. 

This defines a control flow that is orthonogal to the data flow. Whereas
the flow of data is sequential and determined by the data source, the
control flow represents the anytime property as the functions of services
may be called asynchronuous to the data flow.
Figure \ref{fig:control} shows the flow of data and service access.

Examples for services my be classifiers, which provide functions for
predictions based on their current state (model); static lookup
services, which provide additional data to be merged into the stream
or services that can be queried for current statistical information
(mean, average, counts).

\begin{figure}[h!]
  \begin{center}
    \includegraphics[scale=0.35]{graphics/data-control-flow.png}
  \end{center}
  \caption{\label{fig:control}Orthogonal {\em data} and {\em control
      flow}. Processors may use services as well as export
    functionality by providing services.}
\end{figure}

\subsubsection*{Service References and Naming Scheme}
In order to define the data flow as well as the control flow, a naming
scheme is required. Each service needs to have an unique identifier assigned
to it. This identifier is available within the scope of the experiment and
will be used by service consumers (e.g. other processors) to reference that
service.

At a higher level, when multiple experiments or stream environments are running
in parallel, each experiment is associated with an identifier by itself. This
imposes a hierarchical namespace of experiments and services that are defined
within these experiments. The {\em streams} library constitutes a general
naming scheme to allow for referencing services within a single experiment
as well as referring to services within other (running) experiments.

A simple local reference to a service or other element (e.g. a queue) is 
provided by using the identifier (string) that has been specified along with
the service definition. Following a URL like naming format, services within
other experiments can be referenced by using the experiment identifier and
the service/element identifier that is to be referred to within that experiment,
e.g.
\begin{displaymath}
  \mbox{\ttfamily //experiment-3/classifier-2}.
\end{displaymath}
Such names will be used by the \streams library to automatically resolve references
to services and elements like queues.




%
%
%\subsection{Service Registration and Lookup}
%Each service defined within a process needs to have an unique
%identifier assigned to it. This identifier is used to reference that
%service, e.g. from a processor.

%%It is also possible to define standalone services, e.g. for lookup
%%tables on static data. Processors may also consume services.  
%%\subsubsection*{Using Services for Test-then-Train Evaluation}
%A simple example is given by a learning algorithm (classifier). This
%can process data items as part of its learning process. It provides a
%{\em PredictionService}, which contains as single {\ttfamily predict}
%function. This function uses the current prediction model of the
%learning algorithm to return a prediction for a data item. 

%Figure \ref{fig:control} shows the {\em Naive Bayes Learner} embedded
%into a process. This process implements the {\em test-then-train}
%methods for evaluating stream classifiers on labeled data
%streams. Each item of the stream is first used for testing by making a
%prediction for that item, and then used for updating the prediction
%model.

%The first processor {\em Add Prediction} in this process uses the
%prediction service provided by the {\em Naive Bayes Learner} to make
%a prediction for each data item. After the prediction, the item is
%handed over to the learner, which incorporates it into its model.

%After these two processors, the data item contains the original, true
%label, and the prediction added by the {\em Add Prediction} processor.
%The {\em Prediction Error} processor can now apply any loss function
%to determine the prediction error and aggregate that error over time.

%\subsection{Multiple Processes}
%Often, applications require multiple streams and processes to run
%simultaneously. Following this objective, services of processes can be
%accessed from within other processes whereas queues can serve as
%inter-process communication media and synchronization tool.



\newpage
\section{\label{sec:processDesign}Designing Stream Processes}
The former section introduced the main conceptual elements of the
\streams library for creating data flow graphs. Such graphs are
contained within a {\em container}.  A container can be deployed by
compiling the container definition into a data flow graph (or {\em
  compute graph}) for the runtime environment which is to execute the
container.

Each of the basic elements for the process design (i.e. container
definition) directly correspond to an XML element that is used to
define a node in the data flow graph. The following Table
\ref{tab:xmlElements} lists the base elements provided.

\begin{table}[h!]
  \centering{
\renewcommand{\arraystretch}{1.25}
  \begin{tabular}{c|c} \hline
    \bf{Graph Element} & \bf{XML Element} \\ \hline \hline
    Stream & {\ttfamily stream} \\ \hline
    Process & {\ttfamily process} \\ \hline
    Queue  & {\ttfamily queue} \\ \hline
    Service & {\ttfamily service} \\ \hline
  \end{tabular}}
  \caption{\label{tab:xmlElements}The basic XML element used to define a compute graph within the \streams framework.}
\end{table}


%These basic elements are used to define stream processes that can be
%deployed and executed with the \streams runtime environment. The
%stream processes are essentially data flow graphs built from connected
%streams, processes and queues. Such graphs form the basis of a general
%family of message passing frameworks. The \streams framework provides
%a runtime implementation for the deployment and execution of a data
%flow graph.

The definition of stream processes is based on simple XML files that
define processes, streams and queues by XML elements that directly
correspond to the elements presented in Section
\ref{sec:abstraction}. Figure \ref{fig:xmlProcess} shows the scheme of
mapping the XML process definitions into data flow graphs of the
\streams runtime.



In addition there exists mappings for other runtime environments


\subsection{\label{sec:processLayout}Layout of a Process Environment}
As mentioned above, the top-level element of a \streams process
definition is a {\em container}. A single container may contain
multiple processes, streams and services, which are all executed in
parallel. An example for a container definition is provided in
Figure \ref{fig:simpleContainer}.

\begin{figure}[h!]
	\begin{lstlisting}[showstringspaces=false]
      <container id="example">
          <stream id="D" url="file:/test-data.csv" />

          <process input="D">
               <!--
                   The following 'PrintData' is a simple processor that outputs each
                   item to the standard output (console)
                 -->
               <stream.data.PrintData />
          </process>
      </container>
	\end{lstlisting}
	\caption{\label{fig:simpleContainer}A simple container, defining a stream that is created from a CSV file.}
\end{figure}

The core XML elements used in the simple example of Figure
\ref{fig:simpleContainer} are {\ttfamily stream} and {\ttfamily
  process}, which correspond to the conceptual elements that have
previously been introduced in Section \ref{sec:abstraction} and
which are mapped to XML elements according to Table \ref{tab:xmlElements}.
This example defines a container with namespace {\ttfamily example}
which corresponds to the simple compute graph shown in Figure
\ref{fig:simpleGraph}.

\begin{figure}[h!]
  \centering
  \includegraphics[scale=0.3]{graphics/simple-graph}
  \caption{\label{fig:simpleGraph}The simple compute graph that is
    defined by the XML given in Figure \ref{fig:simpleContainer}.}
\end{figure}

The graph contains a single source of data ({\em stream}) and only
one {\em process} element, which consumes the data provided by the
stream and applies the nested processor {\ttfamily PrintData} to
each data item obtained from the stream.

\subsubsection{Defining a Stream Source}
As you can see in the example above, the {\ttfamily stream} element is used to define
a stream object that can further be processed by some processes. The {\ttfamily stream}
element requires an {\ttfamily id} to be specified for referencing that stream as input
for a process. 

In addition, the {\ttfamily url} attribute is used to specify the location
from which the data items should be read by the stream. There exists Java implementations
for a variety of data formats that can be read. Most implementations can also handle 
non-file protocols like {\ttfamily http}. The class to use is picked by the extension
of the URL ({\ttfamily .csv}) or by directly specifying the class name to use:
\begin{figure}[h!]{\footnotesize
    \centering
    \begin{lstlisting}{lang=xml}
       <stream  id="D" class="stream.io.CsvStream"
               url="http://download.jwall.org/stuff/test-data.csv" />
    \end{lstlisting}
    \caption{\label{fig:defStream}Defining a stream that reads from a HTTP resource.}
}
\end{figure}

Additional stream implementations for Arff files, JSON-formatted files or for reading 
from SQL databases are also part of the {\em streams} library. These implementation
also differ in the number of parameters required (e.g. the database driver for SQL
streams). A list of available stream implemenations can be found in Appendix \ref{api:stream:io}.
The default stream implementations also allow for the use of a {\ttfamily limit} parameter
for stopping the stream after a given amount of data items.

\subsubsection{A Stream Process}
The {\ttfamily process} element of an XML definition is associated
with a data stream by its {\ttfamily input} attribute. This references
the stream defined with the corresponding {\ttfamily id}
value. Processes may contain one or more {\em processors}, which are
simple functions applied to each data item as conceptually shown in
\ref{sec:basics}.

A process will be started as a separate thread of work and will read
data items from the associated stream one-by-one until no more data
items can be read (i.e.  {\ttfamily null} is returned by the
stream). Processes in the \streams framework are following greedy
strategy, reading and processing items as fast as possible.

The processes will apply each of the nested processors to the data
items that have been read from the input. The processors will return
a processed data item as result, which is in turn the input for the
next processor embedded into the process. Thus, each process applies
a pipeline of processors to each data item. If any processor of the
pipeline returns {\ttfamily null}, i.e. no resulting data item, then
the pipeline is stopped and the process skips to reading the next
data item from the stream.

The inner processors of a process are generally provided by Java
implementations and are represented by XML elements that reflect
their Java class name. 

In the example in Figure \ref{fig:processXml}, a processor implemented
by the class {\ttfamily my.package.MyProcessor} is added to the
process. The process in this examples is attached to the stream or
queue defined with ID {\ttfamily id-of-input}. Any output of that
process that is not {\ttfamily null}, will be inserted into the queue
with ID {\ttfamily queue-id}. Connecting the output of a process to a
queue (which can then be the input to another processor) is optional.

\begin{figure}[h!]
  \centering
  \begin{lstlisting}
    <process input="id-of-input" output="queue-id">
        <!-- 
             One or more processor elements, referenced
             by their class name, provided with attributes
           -->
        <my.package.MyProcessor param="value" />
    </process>
  \end{lstlisting}
  \caption{\label{fig:processXml}A process references an input
    (i.e. a {\em stream} or a {\em queue}) and contains a list of
    processor elements. Optionally it feeds results to an associated
    output (a {\em queue}).}
\end{figure}

%The general behavior of a process is shown in the pseudo-code
%of Algorithm \ref{alg:process}.
%
%\begin{algorithm}
%\begin{algorithmic}
%\Require{ A data stream $S$ and a sequence $P = \langle f_1,\ldots,f_k\rangle$ of processors}
%\Statex
%\Function{ProcessStream}{$S$}
%   \While{ $true$ }
%      \State{$d :=  \textrm{readNext}( S )$}
%      \ForAll{ $f \in P$ }
%         \State{$d' := f(d)$}
%         \If{$d' = null$}
%         	\Return{$null$}
%         \Else
%	         \State{$d := d'$}
%         \EndIf
%      \EndFor
%   \EndWhile
%\EndFunction
%\end{algorithmic}
%\caption{\label{alg:process}Pseudo-code for the behavior of a simple {\ttfamily process} element.}
%\end{algorithm}

\subsubsection{Processing Data Items}
As mentioned in the previous Section, the elements of a stream are
represented by simple tuples, which are backed by a plain hashmap of
keys to values. These items are the smallest units of data within the
\streams library. 

The smallest {\em functional} units of the \streams library are
provided by simple {\em processor}s.  A {\em processor} is essentially
a function that is applied to a data item and which returns a data
item (or {\ttfamily null}) as result as shown in the {\em identity}
function example below.
\begin{figure}[h!]
   \begin{lstlisting}[language=Java,showstringspaces=false]
	    public Data process( Data item ){
	    	return item;
	    }
   \end{lstlisting}
   \caption{\label{fig:processMethod}The {\ttfamily process(Data)} method - unit of work within {\em streams}.}
\end{figure}

Processors are usually implemented as Java classes, but can be
provided in other (scripting) languages as well. The Java classes are
expected to follow the JavaBeans specification by providing {\ttfamily
  get}- and {\ttfamily set}-methods for properties.  These properties
in turn will be mapped to XML attributes of the corresponding XML
element. This allows processors to be easily provided with parameters
wihtin the XML container definitions.


\subsection{\label{sec:processVariables}Parameterising Containers}
The general structure of container definitions described in Section
\ref{sec:processLayout} allows for the definition of compute graphs
and adding processors and parameters.
For a convenient parameterization, the \streams framework supports the
global definition of properties and includes a intuitive variable
expansion, following the syntax of well known tools like Ant and
Maven.

Variables are specified using the {\ttfamily \$} symbol and curly
bracket wrapped around the property name, e.g. {\ttfamily
  \$\{myVar\}}. This directly allows to access the Java VM system
properties within the container definition. Undefined variables
simple resolve to the empty string.

\begin{figure}[h!]
  \centering
  \begin{lstlisting}
     <container>

         <!-- define property 'baseUrl' using the system property 'user.home'   -->
         <property name="baseUrl" value="file:${user.home}/data/FACT" />

         <stream id="factData" class="fact.io.FactEventStream"
                url="${baseUrl}/example-data.gz" />

         <process input="factData">
            <!-- process the data  -->
         </process>
     </container>
  \end{lstlisting}
  \caption{\label{fig:propertyExample}A container definition using simple variables.}
\end{figure}

As the variable expansion includes the Java system properties, containers
can easily be provided with variables by setting properties when starting
the Java system. The following commands starts the \streams runtime with
a container definition and adds addition variables:

\vspace{1ex}\hspace{2ex}\sample{java -DbaseUrl="/tmp" -cp stream-runner.jar container.xml}

Variables can be used anywhere in the XML attributes, the variables of
a container are expanded at startup time. Therefore any changes of the
variables after the container has been started will not affect the
configuration.


%\input{process-design-distributed}

%\section{\label{sec:distributedProcessing}Distributed Processing}
The concept of containers introduced so far has been focusing on
self-contained containers. One of the reasons for the {\em container}
concept introduced in Section \ref{sec:containers}, is to provide a
possible split of streaming processes into multiple containers, which
can be deployed in a distributed fashion.

\subsection{Naming Scheme and Services}
Two elements are central for distribution: the concept of remote
services and a naming scheme. Both have essentially been introduced
for local elements in Section \ref{sec:stream-api}.

By extending the naming scheme to incorporate the container
identifiers, we extend this to inter-container communication as shown
in Figure 1.

\begin{figure}
  \begin{center}
    \includegraphics[scale=0.5]{graphics/remote-queue.png}
  \end{center}
  \caption{\label{fig:remote-queue}Inter-Container communcation between {\ttfamily crawler} and {\ttfamily storage.}}
\end{figure}


\subsection{Distributing Containers}
In the simplest case, a container is self-contained and will execute
by itself. However, elements within a container may reference elements
in other containers, allowing for a distributed setup of processes.

A very simple example is given by the two containers in Figure 2. The
container {\ttfamily storage} defines a queue and a process that will
store all elements from that queue in a database.

The second container {\ttfamily crawler} reads data items from Twitter
and sends these to the input queue of the {\ttfamily storage}
container.

\begin{figure}
  \begin{center}
    \includegraphics[scale=0.5]{graphics/crawler-storage.png}
  \end{center}
  \caption{\label{fig:crawler-storage} Two simple crawler and storage containers.}
\end{figure}

\subsubsection{Automatic Container Descovery}

By default, each container makes itself available via RMI and responds
to braodcast queries. Therefore no configuration is required as long
as the network infrastructure is capable of distributing the
broadcasts (e.g. in a single ethernet segment).


\subsubsection{Defining Remote Container Connections}

In some situation, the broadcast discover cannot be used or may be
unreliable. To deal with these situations, the naming-service of the
{\em streams} library allows for manually defining references to remote
containers.

The following Figure \ref{fig:container-ref} shows the {\ttfamily
  crawler} container with an explicit RMI reference to the `storage`
container.

\begin{figure}
  \begin{center}
    \includegraphics[scale=0.5]{graphics/crawler-explicit-ref.png}
  \end{center}
  \caption{\label{fig:container-ref}Two simple crawler and storage containers.}
\end{figure}

%
%\section{\label{sec:streamsLibrary}The \streams Library}
The \streams library provides a set of classes and interfaces for the
elements defined in Section \ref{sec:abstraction}, which allows for
implementing custom streams and processors. In addition it provides
basic classes for reading, writing and processing data, e.g. from CSV
files, SVMlight formatted data or by reading streams from an SQL
database. The library consists of three packages:
\begin{enumerate}
\item \textsf{stream-api} -- a small collection of interfaces and classes
  representing the conceptual elements outlined above.
\item \textsf{stream-core} -- several implementations of I/O streams,
  processors, etc. which are of general use.
\item \textsf{stream-runtime} -- a light-weighted execution environment
  that allows to define streaming processes in XML.
\end{enumerate}
To a large extend we focused on developing the \streams API as simple
as possible using standard data structures and following design
patterns and conventions like JavaBeans \cite{javabeans} or techniques
like dependency injection \cite{dependencyInjection} found in well
established frameworks such as the Spring Framework
\cite{springframework} or Google Guice \cite{guice}.

\subsection{Data Items and Processors}
In the \textsf{stream-api} data items are represented by the
{\ttfamily stream.Data} interface, which itself is a plain Java
{\ttfamily Map} with keys of string type and any serializable objects
as values. Maps support our objective to use versatile data structures
that are available and well understood in any language
(e.g. dictionaries in Python or Ruby) and do provide the
self-contained property. The serialization requirement allows data
items to be transferred over network connections, required for running
stream processes in distributed environments.

A data stream is provided by the interface {\ttfamily
  stream.io.DataStream} and basically provides a single {\ttfamily
  readNext()} method returning the next data item of the stream. In
general, the data stream implementations in the \streams library
require a URL or a Java {\ttfamily InputStream} object to read from.
This allows creating streams to read from file, network resources or
from external data generating processes by reading from standard
input.

The processor elements are defined by a simple interface {\ttfamily
  stream.Processor} that requires a single method to be implemented as
shown in Figure \ref{fig:processorInterface}.

\begin{figure}
{\footnotesize
  \begin{lstlisting}
    public interface Processor {
       /**  Method called for each item to be processed.   */
       public Data process( Data item );
    }
  \end{lstlisting}
}
  \caption{\label{fig:processorInterface}Definition of the basic
    processor interface, required to implement custom processors
    within the \streams library.}
\end{figure}
%\begin{figure}[h!]
%
%\begin{verbatim}
%    public interface Processor {
%       /**
%        *  Method called for each item to be processed.
%        */
%       public Data process( Data item );
%    }
%\end{verbatim}
%}
%  \caption{\label{fig:processorInterface}Definition of the basic
%    processor interface, required to implement custom processors
%    within the \streams library.}
%\end{figure}

\subsubsection*{Parameters via JavaBeans}
Following the JavaBeans convention, processors are required to provide
a no-args constructor and may use parameters by simply providing
{\ttfamily get}- and {\ttfamily set}-methods. The example processor
shown in Figure \ref{fig:alertProcessor} (see Appendix) outputs an
alert message for every item that does not provide a ({\em key},{\em
  value}) pair for a user defined key name.  This simple beans
convention allows for automatically registering RapidMiner operators
and their corresponding parameter types. This is provided by the {\em
  RapidMiner-Beans} library.

%\vspace{6cm}

\subsection{Anytime Services}
For implementing the anytime paradigm, the \streams library provides a
{\ttfamily Service} interface and a dynamic naming service which
allows for registering and obtaining services or references to
services. This works similarly to the standard RMI naming services
included in Java, but tries to abstract from a specific
implementation.

The anytime services within the \streams library are implemented by
extending the {\ttfamily Service} interface and defining any method
that shall be provided in an anytime manner. As an example, the
{\ttfamily PredictionService} is implemented by all online learning
algorithms, which defines a simple {\ttfamily predict} method as shown in
Figure \ref{fig:predictionService}. As soon as a processor that
implements a {\ttfamily Service} interface is added to an experiment,
it is automatically registered within the naming service.

\begin{figure}[h!]
\footnotesize
\begin{lstlisting}
   public interface PredictionService extends Service {
       /** Returns the prediction for an item based 
        *  on the current model                      */
       public Serializable predict( Data item );
   }
\end{lstlisting}
\caption{\label{fig:predictionService}A simple {\ttfamily
    PredictionService} that as is provided by all online learning
  algorithms that support classification.}
\end{figure}


%\subsection{Events, Examples and Processors}
%Within RapidMiner, the most fundamental data structure is provided by
%an {\ttfamily IOObject}. Any object exchanged between two operators
%has to implement the {\ttfamily IOObject} interface. From a data
%analysis perspective, these objects typically are sets of examples,
%prediction models, preprocessing models or analysis results such as
%performance vectors.
%
%As the stream processing is mostly concerned with the handling of
%single pieces of data, i.e. single {\em examples} in the RapidMiner
%sense, we denote with {\em data item} the most atomic element of data
%obtained and processed within a stream. A data item is a set of ({\em
%  key},{\em value}) pairs, each of which corresponds to an attribute
%and its value.

\subsection{A light-weight \streams Runtime}
For rapid prototyping and development purposes, the \streams library
implements a small multi-threaded runtime environment, which allows to
define stream processes using a simple XML document. The
interpretation and structure of this XML is very similar to the
notation known from frameworks like Spring \cite{springframework}. A
sample XML process definition of the {\em test-then-train} use case is
provided in the Appendix (Figure \ref{fig:exampleContainer}).
% defines a single stream and a process that
%implements the test-then-train setup mentioned earlier.

The services defined within the \streams API are exported via a naming
service. The default naming service uses a local RMI registry, which
allows for accessing services such as prediction services, or
processors providing meta-data statistics (average, minimum, maximum,
top-k elements) while the processes are running.


%%%
%% Streams Plugin
%%  - Generic Wrapper f�r Stream-API
%%  - Abbildung data-items => IOObject, Processor -> Operator,...
%%  - Stream-IOObject => "file handle", lazy reading/evaluation vs. "lade sofort alles in RAM"
%%  - (continuous) Stream-Process => Sub-Process in RapidMiner
%%  - Anbindung/Integration anderer/alter Operatoren
%%
\section{\label{sec:plugin}RapidMiner Streams-Plugin}
The \plugin\ provides RapidMiner operators for the basic building
blocks of the \streams API using a simple wrapper approach to directly
reuse the processor implementations of most of the \streams
packages. 
%Only a very few special processor have been required to be
%re-written so far. 

The operators of the \plugin\ are automatically created from the
processor and data stream implementations using the \textsf{RapidMiner
  Beans} library. This uses reflection and Java annotations to
automatically extract and set parameter types for the wrapping
operators.
Figure \ref{fig:pluginArchitecture} gives an overview over the
\plugin.  The \streams API serves as an abstraction layer providing
implementations of the basic elements identified in Section
\ref{sec:abstraction}.

\begin{figure}[h!]
\begin{center}
\begin{tikzpicture}[scale=0.75]

  \fill[fill=rapidi!60] (-0.5,6.5) -- (-0.5,2.45) -- (12.5,2.45) -- (12.5,6.5);
  \node at (6,5.75) { \includegraphics[scale=0.4]{graphics/RapidMinerLogo.jpg} };

  \draw[draw=blue!40,fill=blue!18] (0,1.3) -- (12,1.3) -- (12,2.4) -- (0,2.4) -- (0,1.3);
  \node at (6,1.9) { \textsf{Streams API} };
  
  \draw[draw=blue!50,fill=blue!18] (0,0.2) -- (4,0.2) -- (4,1.2) -- (0,1.2) -- (0,0.2) ;
  \node at (2,0.7) { \textsf{Streams Core} };

  \draw[draw=green!50,fill=green!10] (4.1,0.2) -- (8.1,0.2) -- (8.1,1.2) -- (4.1,1.2) -- (4.1,0.2) ;
  \node at (6,0.7) { \textsf{Streams Mining} };

  \draw[draw=green!40,fill=green!10] (8.2,0.2) -- (8.4,0.2) -- (8.4,1) -- (11.8,1) -- (11.8,0.2) -- (12,0.2) -- (12,1.2) -- (8.2,1.2) -- (8.2,0.2) ;

  
  \draw[draw=black!40,fill=black!10]  (8.5,0) -- (8.5,0.9) -- (11.7,0.9) -- (11.7,0);
  \node at (10.1,0.5) {\textsf{MOA}};


  \draw[draw=blue!40,fill=black!14] (0,2.5) -- (4,2.5) -- (4,3.75) -- (10,3.75) -- (10,5) -- (0,5) -- (0,2.5);
  \node at (5,4.3) { \textsf{Streams Plugin} };

  \draw[draw=rapidiText!80,fill=rapidi!18] (4.1,2.5) -- (12,2.5) -- (12,5) -- (10.1,5) -- (10.1,3.65) -- (4.1,3.65) -- (4.1,2.5);
  \node[color=rapidiText!80] at (9,3.05) { \textsf{RapidMiner Beans} };


\end{tikzpicture}
\end{center}
\caption{\label{fig:pluginArchitecture}The architecture of the
  \textsf{Streams Plugin}, built on top of the \streams API. The
  packages marked as green are work in progress and have not yet been
  fully integrated.}
\end{figure}

\subsection{A Stream Process within RapidMiner}
The elements for {\em streams} and {\em processors} are represented by
RapidMiner operators. The continues {\em process} is mapped to a
RapidMiner subprocess. Figure \ref{fig:rapidMinerStream} shows a
stream process within RapidMiner.

\begin{figure}[h!]
  \begin{center}
    \includegraphics[scale=0.4]{graphics/StreamProcess.png}
  \end{center}
  \caption{\label{fig:rapidMinerStream}A continuous stream process in
    RapidMiner. The top process shows a stream operator and the stream
    process as a subprocess. The edge \pointer{1} represents a data
    stream object. Within the stream subprocess, IOObjects transported
    are single data items (edge \pointer{2}).}
\end{figure}


\subsection{Control Flow and Anytime Services}
Operators that relate to processors implementing a {\ttfamily Service}
interface will be automatically registered as services within a
RapidMiner naming service provided by the \plugin. They can be
referenced by consuming operators using a simple drop-down select box
within the operator parameter view of RapidMiner.

For accessing services or monitoring from outside the continous
streaming process, the \plugin\ integrates an embedded web server that
exports services via a web service interface. Currently services are
exported via this embedded web server using the simple JSON-RPC
protocol \cite{jsonRPC} and a local RMI registry.

%\section{``Power of Abstraction'''}
%
%\section{The RapidMiner Data Stream Plugin}
%\subsection{Architecture}

%\newpage
%
%Documentation with markdown
%
%\begin{figure}
%  \begin{center}
%    \includegraphics[scale=0.4]{doc.png}
%  \end{center}
%\end{figure}

\clearpage
\newpage

\section{\label{sec:applications}Example Applications}
In this section we will give a more detailed walk-through of some
applications and use-cases that the \streams library is used for.
These examples come from various domains, such as pre-processing
of scientific data, log-file processing or online-learning by
integrating the MOA library.

Most of the use-cases require additional classes for reading and
processing streams, e.g. stream implementations for parsing
domain-specific data formats. Due to the modularity of the
\streams library, domain-specific code can easily be
added and directly used within the process design.

\subsection{\label{sec:fact}FACT Data Analysis}
The first use-case we focus on is data pre-processing in the domain of
scientific data obtained from a radio telescope. The FACT project
maintains a telescope for recording cosmic showers in a fine grained
resolution. The telescope consists of a mirror area which has a
1440-pixel camera mounted on top as shown in Figure
\ref{fig:factTelescope}. This camera is recording electric
energy-impulses which in turn is measured by sampling each pixel at a
rate of 2 GHz.

\begin{figure}[h!]
\centering
\includegraphics[scale=0.2]{graphics/fact-telescope.jpg}
\caption{\label{fig:factTelescope}The FACT telescope on La Palma.}
\end{figure}

Based on trigger-signals, small sequences (a few
nanoseconds) of these energy-impulses are recorded and stored into 
files. Each sequence is regarded as a single {\em event}. 
The electronics of the telescope are capable of recording about
60 events per second, resulting in a data volume of up to 10 GB
of raw data that is recorded over a 5 minute runtime. The captured
data of those 5-minute runs is stored in files.
% using the FITS format.

%\bigskip

The long-term objective of analyzing the FACT data comprises several tasks:
\begin{enumerate}
  \item Identify events that represent showers.
  \item Classify these events as Gamma or Hadron showers.
  \item Use the Gamma events for further analysis with regard to physics stuff.
\end{enumerate}

Besides the pure analysis the data has to be pre-processed in order
to filter out noisy events, calibrate the data according to the state
of the telescope electronic (e.g. stratify voltages over all camera pixels).

%As can be seen in Figure \ref{fig:eventStream}, the basic view of the 
%FACT data is that of a continuous stream of events. The plugin provides
%a FACT-stream operator that allows to open a (compressed) FACT file in FITS format
%and also implements the DRS calibration used to calibrate the RAW
%data. The outcome is a stream of calibrated event objects.

%A {\em DataStreamProcessing} operator can then be attached to that
%stream to iteratively process the stream event-by-event. That way,
%only a single event is read into main memory at a time.

%\subsubsection{Processing Data of the FACT Telescope}
%
%This raw data needs an additional calibration before any furthr analysis
%can be applied. An abstract outline of the FACT analysis is shown in
%Figure \ref{fig:eventStream}.

\begin{figure}[h!]
\begin{center}
\includegraphics[scale=0.2]{fact-event-stream}
\end{center}
\caption{\label{fig:eventStream}Stream-lined processing of events.}
\end{figure}





\subsubsection{Reading FACT Data}
The {\em fact-tools} library is an extension of the \streams framework that
adds domain specific implementations such as stream-sources and specific 
processors to process FACT data stored in FITS files. These processors provide
the code for calibrating the data according to previously observed parameters,
allow for camera image analysis (image cleaning) or for extracting features
for subsequent analysis.

The XML snippet in Figure \ref{fig:readFACTxml} defines a simple process to
read raw data from a FITS file and apply a calibration step to transform that
data into correct values based upon previously recorded calibration parameters.

\begin{figure}[h!]
\begin{lstlisting}{language=XML}
    <container>
        <stream id="factData" url="file:/data/2011-09-13-004.fits.gz"
             class="fact.io.FACTEventStream" />

        <process input="factData">
             <fact.io.DrsCalibration file="/data/2011-09-13-001.fits.drs.gz" />
             <!--  add further processors here  -->
        </process>
    </container>
\end{lstlisting}
\caption{\label{fig:readFACTxml}Basic process definition for reading raw FACT data
and calibrating that data.}
\end{figure}

Each single event that is read from the event stream, contains the
full raw, calibrated measurements of the telescope. 
%As with all other
%data stream implementations, the {\em FactEventStream} emits a sequence
%of {\em data items}, each of which contains the data of a single FACT
%event.
The attributes of the data items reflect the image data, event
meta information and all other data that has been recorded during the
observation. Table \ref{tab:factEventKeys} lists all the attributes of
an event that are currently provided by the {\ttfamily FACTEventStream}
class.
\begin{table}[h!]
\renewcommand{\arraystretch}{1.25}
{\footnotesize
  \begin{center}
    \begin{tabular}{l|l} \hline
      \textsf{\textbf{Name (key)}} & \textsf{\textbf{Description}} \\ \hline \hline
      {\ttfamily EventNum} & The event number in the stream \\ \hline
      {\ttfamily TriggerNum} & The trigger number in the stream \\ \hline
      {\ttfamily TriggerType} & The trigger type that caused recording of the event \\ \hline
      {\ttfamily NumBoards} & \\ \hline
      {\ttfamily Errors} & \\ \hline
      {\ttfamily SoftTrig} & \\ \hline
      {\ttfamily UnixTimeUTC} & \\ \hline
      {\ttfamily BoardTime} & \\ \hline
      {\ttfamily StartCellData} & \\ \hline
      {\ttfamily StartCellTimeMarker} & \\ \hline
      {\ttfamily Data} & The raw data array ($1440\cdot 300 = 432000$ float values) \\ \hline
      {\ttfamily TimeMarker} & \\ \hline
      {\ttfamily @id} & A simple identifier providing date, run and event IDs \\ \hline
      {\ttfamily @source} & The file or URL the event has been read from \\ \hline
    \end{tabular}
  \end{center}}
  \caption{\label{tab:factEventKeys} The elements available for each event.}
\end{table}

The {\ttfamily @id} and {\ttfamily @source} attributes provide
meta-information that is added by the FACT-stream implementation
itself, all the other attributes are provided within the FITS data
files. The {\ttfamily @id} attribute's value is created from the
{\ttfamily EventNum} and date when the event was recorded,
e.g. {\ttfamily 2011/11/27/42/8}, denoting the event 8 in run 42 on
the 27th of November 2011.


\subsubsection{Processors for FACT Data}
Any of the existing core processors of the \streams library can directly be applied
to data items of the FACT event stream. This already allows for general applications
such as adding additional data (e.g. whether data from a database).

The {\em fact-tools} library provides several domain specific processors that focus
on the handling of FACT events. The {\ttfamily DrsCalibration} processor for calibrating
the raw data has already been mentioned above.

Other processors included are more specifically addressing the image-analysis task:
\begin{itemize}
  \item {\ttfamily fact.data.CutSlices} \\
  Which can be used to select a subset of the raw data array for only a excerpt of
  the region-of-interest\footnote{The {\em region of interest} is the length of the
  recorded time for each event, usually 300 nanoseconds, at most 1024 nanoseconds} (ROI).

  \item {\ttfamily fact.data.SliceNormalization} \\
  As there is a single-valued series of floats provided for each pixel, this processor
  allows for normalizing the values for these series to $[0,1]$.

  \item {\ttfamily fact.data.MaxAmplitude} \\
  This processor extracts a float-array of length 1440, which contains the maximum
  amplitude for each pixel.

  \item {\ttfamily fact.image.DetectCorePixel} \\
  This class implements a heuristic strategy to select the possible core-pixels of
  a shower, that may be contained within the event.
\end{itemize}


\subsection*{Installing the FACT-Plugin}
The FACT-Plugin is an extension for RapidMiner. The RapidMiner
open-source software is written in Java and is available for multiple
environments (Windows, Unix). It can be downloaded from
\url{http://rapid-i.com}.

The FACT-Plugin is a simple Java archive ({\em jar}-file) that can be
found at
\begin{displaymath}
 \mbox{\url{http://sfb876.tu-dortmund.de/FACT/}}
\end{displaymath}
To install the plugin simply copy the latest {\ttfamily FACT-Plugin.jar}
to your RapidMiner {\ttfamily lib/plugins} directory. After restarting
RapidMiner, the plugin will be loaded and all of its additional operators
will show up in the operators list.





%%%
%%
%%
\subsection{\label{sec:videoStreams}Processing Video Streams}
\baustelle The processing of video streams is a recent application
that has been added to the \streams framework. The {\em streams-video}
package provides an implementation for streams of video frames in the
MJPEG format.

Several procssors have been added, e.g. for the extraction of features
from online video data. Features from videos might e.g. be indicating
commercial breaks or switches in scenes. 
cannot be derived from static EPG data or any other annotating data
sources. An overview about the video processing and the associated
scene and shot detection is outlined in Section
\ref{sec:videoFeatures}.

Researching the feature extraction from video data focuses on image
analysis of single video frames and the derived stream of features from
a series of such frames. The static feature extraction from images is
being investigated with the RapidMiner Image Mining Plugin \cite{Burget2010a}.

To support the online feature extraction from video data, the TUDO
group provides an image stream implemented for the \streams framework
which allows for applying processors to video frames in a streaming
manner. A sample of process definition for reading RAW video streams
is shown in Figure \ref{fig:videoXml}.

\begin{figure}[h!]
  \centering
  \begin{lstlisting}
    <container>

        <!--  This stream reads encoded JPG images from a file       -->    
        <stream  id="video" class="stream.io.MJpegImageStream"
                url="file:/Volumes/RamDisk/tagesschau.mjpeg.stream" />

        <process input="video">
             <!--  Add a 'frame:id' attribute, counting the frames    -->
             <CreateID key="frame:id" />
        
             <!--  Extract the average RGB channel values             -->
             <vista.video.ExtractAverageRGB />

             <!--  Plot the average red/green/blue channel values     -->         
             <WithKeys keys="frame:*" >
                  <stream.plotter.Plotter keys="frame:red:avg,frame:blue:avg,frame:green:avg"
                                       history="1000" />
             </WithKeys>
        </process>
    </container>
  \end{lstlisting}
  \caption{\label{fig:videoXml}A process definition for reading RAW
    video files encoded as a sequence of JPEG images. The processors
    used in this sample allow for extracting the average R/G/B channel
    values for each frame. The result is a 3-dimensional time series,
    emitting data at the usual frame rate of 25 samples per second.}
\end{figure}

\subsubsection*{Decoding Videos to MJPEG}
The format required for the {\ttfamily MJpegImageStream}
implementation to read video frames is a file of RAW decoded video
frames in JPEG image format. The stream implementation does not do the
decoding itself. Such a decoding can easily be derived from the open
source {\em MPlayer} tool, which is available for Linux, MacOS and
Windows. The {\em MPlayer} tool provides a standalone video decoder
called {\ttfamily mencoder} that can decode video streams and encode
these into various formats, including RAW JPEG sequence files.

For decoding a file {\ttfamily tagesschau.mp4}\footnote{The {\em
    Tagesschau} is a German news show, which provides hourly summaries
  of 90 seconds length. These can be downloaded from
  \url{http://www.tagesschau.de} and are used here for demonstration
  only.}, the {\ttfamily mencoder} needs to be invoked as:
\begin{verbatim}
   # mencoder tagesschau.mp4 -nosound -of rawvideo -o tagesschau.raw \
        -ovc lavc -lavcopts vcodec=mjpeg
\end{verbatim}
This will write a RAW JPEG sequence file to {\ttfamily tagesschau.raw}
which in turn can be read with the {\ttfamily MJpegImageStream} as shown
in Figure \ref{fig:videoXml}.

The {\ttfamily MJpegImageStream} will provide data items that contain
the raw byte array of the JPEG image in the {\ttfamily image:data}
attribute. This can further be used by processors to decode the original
bitmap image or display the image. The processor {\ttfamily AverageRGB}
decodes the image to a bitmap in order to compute the average color
channel values for the complete frame.

The {\ttfamily mencoder} tool as well as the \streams framework support
reading from standard input and writing to standard output. This allows
for decoding and processing live video streams from the Zattoo platform
later on.


\begin{figure}[h!]
  \centering
  \includegraphics[scale=0.3]{graphics/video-stream.png}
  \caption{\label{fig:videoStreaming}The \streams framework
    continuously extracting average R/G/B channel values from a sample
    video while displaying the video frames to the screen.}
\end{figure}

%\subsection{Analyzing Log-File Streams}
%
%\subsection{Monitoring Steel-Production Processes}
%%%
%% Use-Case Examples
%%   * Beispiel f�r Continuous Processing:
%%      - Test-then-train (Naive Bayes)
%%      - Meta-Data Statistics (anytime service via Web-Interface+RMI)
%%      - Live-Plotter (z.B. Prediction Error over time...)
%%
%%   * Beispiel Prozesse f�r Mini-Batch processing
%%      - FACT Szenario: 1 data item => 1 ExampleSet
%%      - Mini-Batch zur Evaluation (Average Prediction Performance 
%%        auf 27 GB CSV-Datei mit nur "ganz wenig" Speicher)
%%
\section{\label{sec:usecases}Examples: Online Processing with RapidMiner}
As mentioned in the introduction, we now provide some examples on
how to process data within the \textsf{DataStream Plugin}.

%\begin{appendix}
\section{The \streams Core Classes}
The \streams framework provides a wide range of implementations for
data streams and processors. These are useful for reading application
data and defining a complete data flow.

In this section we provide a comprehensive overview of the classes and
implementations already available in the \streams library. These can
directly be used to design stream processes for various application
domains.

\subsection{Data Streams}

\subsection{Data Processors}

\subsubsection{Processors in {\ttfamily stream.flow}}

\Package{stream.flow}
The \texttt{stream.flow} package contains processors that allow for data
flow control within a process setup. Processors in this package are
usually processor-lists, i.e.~they may provide nested processors that
are executed based on conditions.

A typical example for control flow is given with the following
\texttt{If} processor, which executes the \texttt{PrintData} processor
only, if the value of attribute \texttt{x1} is larger than 0.5.
\begin{figure}[h!]
  \begin{lstlisting}{lang=xml}
        <If condition="\%{data.x1} @gt 0.5">
          <PrintData />
        </If>
  \end{lstlisting}
\end{figure}
Other flow control processors provide control of data queues such as
enqueuing events into other processes' queues.


%\input{stream_flow_CreateAndEnqueue}
%\input{stream_flow_CreateAndMultiEnqueue}
\input{stream_flow_Delay}
\input{stream_flow_Enqueue}
\input{stream_flow_Every}
\input{stream_flow_If}
\input{stream_flow_OnChange}
\input{stream_flow_Skip}
\input{stream_flow_Collect}
\input{stream_flow_ForEach}

\subsubsection{Processors in {\ttfamily stream.data}}

\subsubsection{Processors in {\ttfamily stream.parser}}

\section{The \streams Runtime}
Along with the \streams API, that is provided for implementing custom
streams or processors, the \streams framework provides a runtime
environment for running stream containers.

\subsection{Installing the \streams Runtime on Debian/RedHat}
For Debian and RPM based systems, there exists a package repository,
that provides Debian and RPM packages that can easily be installed
using the system's package managers. A step-by-step guide for setting
up the package manager on Debian and Ubuntu systems is provided in
Section \ref{sec:installDeb}. Instructions for RedHat based systems
such as RedHat, CentOS or Scientific Linux are provided in
\ref{sec:installRPM}.


\subsubsection{\label{sec:gpgKey}Signatures for Packages}
The repositories and all packages within the repository are
cryptographically signed with a GPG key with ID {\ttfamily 0x13443F4A}
to ensure their consistency. The key is available at

\sample{http://download.jwall.org/software.gpg}

The key is associated with the following information:

\sample{User ID: jwall.org Software Repository <software@jwall.org>\newline
Fingerprint: 175C 915F 51CA 8AA2 387B  E3E8 48E6 B98D 1344 3F4A}

This key needs to be added to the package management key ring of the
system (e.g. {\em apt} on Debian or {\em yum} on RedHat systems).

\subsubsection{\label{sec:installDeb}Installing on Debian/Ubuntu}
There exists a Debian/Ubuntu repository at {\ttfamily
  jwall.org}\footnote{The site \url{http://www.jwall.org/streams/} is
  the base web-site of the \streams framework.} which provides access
to the latest release versions of the \streams library.

To access this repository from within your Debian system, you'll need
create a new file {\ttfamily /etc/apt/sources.list.d/jwall.list} with
the following content:

\sample{ deb http://download.jwall.org/debian/ jwall main } 

The repositories and all packages within the repository are
cryptographically signed with a GPG key. Please see Section
\ref{sec:gpgKey} above for details on how to verify the correctness of
this key.

This key needs to be added to the APT key ring of the Debian/Ubuntu
system by running the following commands (the {\ttfamily \#} denotes
the shell prompt):

\sample{\# sudo wget http://download.jwall.org/debian/software.gpg\newline
\# sudo apt-key add software.gpg 
}

After the key and the repository have been added to the APT package
management, all that is left is to update the package list and install
the \streams environment with the following commands:

\sample{\# sudo apt-get update\newline
\# sudo apt-get install streams
}

The first command will update the package lists, the second will install
the lastest version of the {\ttfamily streams} package. After installation,
the system should be equipped with a new {\ttfamily stream.run} command
to run XML stream processes:

\sample{\# stream.run my-process.xml}

\subsubsection{\label{sec:installRPM}Installing on RedHat/CentOS/Fedora}
There exists a YUM repository at the {\ttfamily jwall.org} site, which
provides access to the latest release versions of the \streams framework
for RedHat based systems.

To access this repository from within your CentOS/RedHat system,
you'll need to create a file {\ttfamily /etc/yum.repos.d/jwall.repo}
with the following contents:

\sample{[jwall]\newline
name=CentOS-jwall - jwall.org packages for noarch\newline
baseurl=http://download.jwall.org/yum/jwall\newline
enabled=1\newline
gpgcheck=1\newline
protect=1}

The RPM packages are signed with a GPG key, please see Section
\ref{sec:gpgKey} for information how to validate this key.

To import the GPG key into your system's key ring, run the
following command as super user:

\sample{\# rpm --import http://download.jwall.org/software.gpg}

After the key has been imported your system is ready to install
the \streams package using the system's package manager, e.g.
by running

\sample{\# yum install streams}

This will download the required packages and set up the system
to provide the {\ttfamily stream.run} command to execute XML
stream processes.
%\section{\label{sec:apiReference}API Reference}

%\subsubsection{Package {\ttfamily stream.io}}
\subsection{\label{app:dataStreams}\label{api:stream:io}Data Stream Implementations}

Reading data is usually the first step in data processing. The package
{\ttfamily stream.io} provides a set of data stream implementations
for data files/resources in various formats.

All of the streams provided by this package do read from URLs, which
allows reading from files as well as from network URLs such as HTTP
urls or plain input streams (e.g. standard input).

The streams provide an iterative access to the data and use the default
\texttt{DataFactory} for creating data. They do usually share some
common parameters supported by most of the streams such as
\texttt{limit} or \texttt{username} and \texttt{password}.

\subsubsection*{Defining a Stream}
As discussed in Section \ref{sec:designingProcesses}, a stream is
defined within a container using the XML {\ttfamily stream} element,
providing a {\ttfamily url} and {\ttfamily class} attribute which
determines the source to read from and the class that should be used
for reading from that source. In addition, the definition requires a
third attribute {\ttfamily id}, which assigns the stream with a
(unique) identifier. This identifier is then used to reference the
stream as input to a process.

As a simple example, the following XML snippet defines a data stream
that reads data items in CSV format from some file URL:
\begin{figure}[h!]
        \centering
        \begin{lstlisting}{lang=xml}
           <stream  id="csv-data" class="stream.io.CsvStream"
                   url="file:/tmp/example.csv" />
        \end{lstlisting}
        \caption{Defining a CSV stream from a file.}
\end{figure}

\subsubsection*{Streaming Data from various URLs}
The \streams runtime supports a list of different URL schemes which
are provided by all Java virtual machines, e.g. {\ttfamily http} URLs
or {\ttfamily file} URLs. Custom URL schemes can also be registered
within the Java VM. As of this, the \streams runtime additionally
offers a {\ttfamily classpath:} and a {\ttfamily system:} URL scheme.

The {\ttfamily classpath:} URLs can be used to create data streams
that read from resources which are available on the classpath. This is
useful for providing example sources within custom JAR files or the
like. The following example shows how to create a stream that reads
data in JSON format from a resource {\ttfamily example.json} that is
searched for in the default classpath:
\begin{figure}[h!]
        \centering
        \begin{lstlisting}{lang=xml}
           <stream  id="json-stream"  class="stream.io.JSONStream"
                   url="classpath:/example.json" />
        \end{lstlisting}
        \caption{\label{fig:jsonStreamClasspath}Defining a JSON stream from a classpath resource.}
\end{figure}

To support streams that read data from standard input or standard
error, the library provides the {\ttfamily system:} URL schema. This
schema provides access to the system input and error streams and are
useful when piping data to a stream via the command line, e.g. by
running a command like:
\begin{figure}[h!]
\sample{\# cat data.csv | stream.run my-process.xml}
\end{figure}
To define a stream that reads from standard input, simply specify
{\ttfamily system:input} as the streams URL as shown in figure
\begin{figure}[h1]
        \centering
        \begin{lstlisting}{lang=xml}
           <stream  id="example"  class="stream.io.CsvStream"
                   url="system:input" />
        \end{lstlisting}
        \caption{\label{fig:csvStreamStdin}Defining a CSV stream that reads data from the system's standard input.}
\end{figure}

\newpage
\input{stream_io_ArffStream}

\input{stream_io_CsvStream}

\input{stream_io_JSONStream}
\input{stream_io_LineStream}
\input{stream_io_ProcessStream}
\input{stream_io_SQLStream}
\input{stream_io_SvmLightStream}
\input{stream_io_TimeStream}

\newpage
\subsection{\label{api:stream:queues}Queue Implementations}
The notion of queues is similar to the definition of streams within
the \streams framework. Queues provide can be attached as sources to
processes while also allowing to be fed with data items from other
places. This allows for simple inter-process communication by
forwarding data items from one process to the queue that is read by
another different process.



\input{stream_io_BlockingQueue}
%\input{stream_io_CsvUpload}
%\input{stream_io_CsvWriter}
%\input{stream_io_JSONWriter}
%\input{stream_io_LineWriter}
%\input{stream_io_ListDataStream}
%\input{stream_io_SQLWriter}
%\input{stream_io_SvmLightStream}
%\input{stream_io_SvmLightWriter}
%\input{stream_io_TimeStream}
%\input{stream_io_TreeStream}


\end{appendix}

%\section{\label{sec:Outlook}Conclusion and Future Work}
In this work we presented a simple abstraction API for modelling
continuous streaming processes and implementing custom processors and
services. On top of this layer of abstraction we implemented a
RapidMiner \plugin\  that integrates the stream oriented processing into
the RapidMiner suite. This allows for processing of continuous data or
large batch data sets using sequential single item or mini batch
processing.
%
%The API is simple to extend and new operators can easily be integrated
%by following ...
%
%\medskip

Future work will focus on integrating MOA into the \streams library,
making more data mining algorithms available for online processing. In
addition we seek for extending the remote access for other web service
protocols like SOAP. An interesting extension will be the integration
of distribution capabilities, e.g. by incorporating support for
backend infrastuctures like Twitter's {\em Storm} framework.

\paragraph{Acknowledgements} This work was supported by the DFG within
the Collaborative Research Center on {\em Providing Information by
  Resource-Constrained Data Analysis} (SFB-876).



% Bibliography using Bibtex, style plain
\bibliographystyle{plain}
\bibliography{literatur,local-refs}


\end{document}
