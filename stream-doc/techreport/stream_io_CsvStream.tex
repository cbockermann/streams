\subsubsection{CSVStream}

This data stream source reads simple comma separated values from a
file/url. Each line is split using a separator (regular expression).

Lines starting with a hash character (\texttt{\#}) are regarded to be
headers which define the names of the columns.

The default split expression is \texttt{(;\textbar{},)}, but this can
changed to whatever is required using the \texttt{separator} parameter.

\begin{figure}[h]
\begin{center}{\footnotesize
{\renewcommand{\arraystretch}{1.4}
\textsf{
\begin{tabular}{|c|c|p{9cm}|c|} \hline
\textbf{Parameter} & \textbf{Type} & \textbf{Description} & \textbf{Required} \\ \hline  
{\ttfamily keys } & String[] &  & ? \\ \hline
{\ttfamily separator } & String &  & true\\ \hline
{\ttfamily id } & String &  & ? \\ \hline
{\ttfamily password } & String & The password for the stream URL (see username parameter) & false\\ \hline
{\ttfamily prefix } & String & An optional prefix string to prepend to all attribute names & false\\ \hline
{\ttfamily limit } & Long & The maximum number of items that this stream should deliver & false\\ \hline
{\ttfamily username } & String & The username required to connect to the stream URL (e.g web-user, database user) & false\\ \hline
\end{tabular}
 } 
 } 
 } 
\caption{Parameters of class {\ttfamily stream.io.CsvStream}}
\end{center}
\end{figure}
