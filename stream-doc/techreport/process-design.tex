\section{\label{sec:processDesign}Designing Stream Processes}
The former section introduced the main conceptual elements of the \streams library.
These basic elements are used to define stream processes that can be deployed and
executed with the \streams runtime environment.

The definition of stream processes is based on simple XML files that define processes,
streams and queues by XML elements that directly correspond to the elements presented
in Section \ref{sec:abstraction}.


\subsection{Layout of a Process Environment}
The \streams library follows the concept of runtime containers, known from Java's
servlet specification and similar architectures. A single container may contain
multiple processes, streams and services, which are all executed in parallel. A
contains is simply providing the abstract environment and a joint namespace for
these elements to execute.

An example for a container definition is provided in Figure \ref{fig:simpleContainer}.
This example defines a process environment with namespace {\ttfamily example} which
contains a single data stream with identifier {\ttfamily D} and a process that will
be processing that stream.

\begin{figure}[h!]
	\begin{lstlisting}[showstringspaces=false]
      <container id="example">
          <stream id="D" url="file:/test-data.csv" />

          <process input="D">
               <!--
                   The following 'PrintData' is a simple processor that outputs each
                   item to the standard output (console)
                 -->
               <stream.data.PrintData />
          </process>
      </container>
	\end{lstlisting}
	\caption{\label{fig:simpleContainer}A simple container, defining a stream that is created from a CSV file.}
\end{figure}

The core XML elements used in the simple example of Figure \ref{fig:simpleContainer}
are {\ttfamily stream} and {\ttfamily process}, which correspond to the same 
conceptual elements that have previously been defined in Section \ref{sec:abstraction}.

\subsubsection{Defining a Stream Source}
As you can see in the example above, the {\ttfamily stream} element is used to define
a stream object that can further be processed by some processes. The {\ttfamily stream}
element requires an {\ttfamily id} to be specified for referencing that stream as input
for a process. 

In addition, the {\ttfamily url} attribute is used to specify the location
from which the data items should be read by the stream. There exists Java implementations
for a variety of data formats that can be read. Most implementations can also handle 
non-file protocols like {\ttfamily http}. The class to use is picked by the extension
of the URL ({\ttfamily .csv}) or by directly specifying the class name to use:
\begin{figure}[h!]{\footnotesize
    \centering
    \begin{lstlisting}{lang=xml}
       <stream  id="D" class="stream.io.CsvStream"
               url="http://download.jwall.org/stuff/test-data.csv" />
    \end{lstlisting}
    \caption{\label{fig:defStream}Defining a stream that reads from a HTTP resource.}
}
\end{figure}

Additional stream implementations for Arff files, JSON-formatted files or for reading 
from SQL databases are also part of the {\em streams} library. These implementation
also differ in the number of parameters required (e.g. the database driver for SQL
streams). A list of available stream implemenations can be found in Appendix \ref{api:stream:io}.
The default stream implementations also allow for the use of a {\ttfamily limit} parameter
for stopping the stream after a given amount of data items.

\subsubsection{A Stream Process}
The {\ttfamily process} element of an XML definition is associated with a data stream
by its {\ttfamily input} attribute. This references the stream defined with the corresponding
{\ttfamily id} value. Processes may contain one or more {\em processors}, which are simple
functions applied to each data item as conceptually shown in \ref{sec:basics}.

A process will be started as a separate thread of work and will read data items from
the associated stream one-by-one until no more data items can be read (i.e. 
{\ttfamily null} is returned by the stream). The general behavior of a process is
shown in the pseudo-code of Algorithm \ref{alg:process}.

\begin{algorithm}
\begin{algorithmic}
\Require{ A data stream $S$ and a sequence $P = \langle f_1,\ldots,f_k\rangle$ of processors}
\Statex
\Function{ProcessStream}{$S$}
   \While{ $true$ }
      \State{$d :=  \textrm{readNext}( S )$}
      \ForAll{ $f \in P$ }
         \State{$d' := f(d)$}
         \If{$d' = null$}
         	\Return{$null$}
         \Else
	         \State{$d := d'$}
         \EndIf
      \EndFor
   \EndWhile
\EndFunction
\end{algorithmic}
\caption{\label{alg:process}Pseudo-code for the behavior of a simple {\ttfamily process} element.}
\end{algorithm}

\subsubsection{Processing Data Items}
As mentioned in the previous Section, the elements of a stream are represented by simple
tuples, which are backed by a plain hashmap of keys to values. These items are the smallest
units of data within the \streams library. They are read from the stream by the process and 
handled as shown in Algorithm \ref{alg:process}. 

The smallest unit of work that the \streams library defines is a simple {\em processor}.
A {\em processor} is essentially a function that acts upon a data item. The processors
already available in the library are provided by Java classes, which implement the
simple {\ttfamily stream.Processor} interface. The interface defines a single function as
shown in Figure \ref{fig:processFunction}.
\begin{figure}[h!]
	\begin{lstlisting}[language=Java,showstringspaces=false]
	public interface Processor {
	    public Data process( Data item ){
	    	return item;
	    }
	}
	\end{lstlisting}
	\caption{\label{fig:processFunction} The {\ttfamily Processor} interface that needs to be implemeted to create new processor elements.}
	\end{figure}