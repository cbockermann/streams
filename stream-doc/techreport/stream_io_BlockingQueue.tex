\StreamSection{BlockingQueue}

The class {\ttfamily stream.io.BlockingQueue} provides a simple
DataStream that items can be enqueued into and read from. This allows
inter-process communication between multiple active processes to be
designed using data items as messages.

As the name already suggests, this queue is a blocking queue,
resulting in any process that reads from this queue to block if the
queue is empty. Likewise, any processor that adds items to the
queue (e.g. {\ttfamily stream.flow.Enqueue}) will be blocking if
the queue is full.

By default the size of the queue is unbounded (i.e. bound by the
available memory only), but can be fixed by using the {\ttfamily size}
parameter.

\begin{table}[h]
\begin{center}{\footnotesize
{\renewcommand{\arraystretch}{1.4}
\textsf{
\begin{tabular}{|c|c|p{9cm}|c|} \hline
\textbf{Parameter} & \textbf{Type} & \textbf{Description} & \textbf{Required} \\ \hline  
{\ttfamily size } & int &  & ? \\ \hline
{\ttfamily id } & String & The ID of this string for associating it with processes. & true\\ \hline
{\ttfamily password } & String & The password for the stream URL (see username parameter) & false\\ \hline
{\ttfamily prefix } & String & An optional prefix string to prepend to all attribute names. & false\\ \hline
{\ttfamily limit } & Long & The maximum number of items that this stream should deliver. & false\\ \hline
{\ttfamily username } & String & The username required to connect to the stream URL (e.g web-user, database user) & false\\ \hline
\end{tabular}
 } 
 } 
 } 
\caption{Parameters of class {\ttfamily stream.io.BlockingQueue}.}
\end{center}
\end{table}
