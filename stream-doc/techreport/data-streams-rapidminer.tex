
\subsection*{Data Streams in RapidMiner}
The focus of RapidMiner so far has been batch processing. With Radoop,
this has been extended to (massive) parallel partial batch processing
on top of the Hadoop clustering software. The \streams library
proposed in this work aims at providing {\em continuous stream
  processing} of non-stationary data. Based on this, we provide a
\plugin\ for RapidMiner to extend its processing capabilities to the
continuous stream setting. Table \ref{tab:software} summarizes the
mentioned software libraries with respect to their focused processing
mode.

%In contrast to existing libraries, we seek to find an abstraction
%layer to model and develop stream processing algorithms while being
%able to map these algorithms onto different backend execution
%engines. 
The \streams library provides a simple execution runtime by itself
whereas the \plugin\ implements an execution environment within
RapidMiner, making the implemented algorithms available in the
RapidMiner suite. However, the level of abstraction provided by the
\streams does not limit the execution of stream processes to the
\streams runtime, but also aims at including large scale distributed
execution environments (e.g.~\textsf{Storm}).

\begin{table}
  \begin{center}
{
\linespread{1.4}
\textsf{
    \begin{tabular}{c|c}
      \textbf{Processing Model} & \textbf{Supporting Software} (examplary) \\ \hline
      Batch Processing & WEKA, {\color{blue} RapidMiner} \\ \hline
      Parallel Batch Processing & Google MapReduce, Hadoop, {\color{blue} Radoop} \\ \hline
      Continuous Stream Processing & S4, Storm, MOA, {\color{blue} \plugin}
    \end{tabular}
}
}
    \caption{\label{tab:software}Different software frameworks for
      different processing models. Packages/libraries marked as blue
      are related to RapidMiner (extensions, plugins).}
  \end{center}
\end{table}
