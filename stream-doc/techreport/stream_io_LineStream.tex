\subsubsection{LineStream}

This is a very simple stream that just reads from a URL line-by-line.
The content of the line is stored in the attribute determined by the
\texttt{key} parameter. By default the key \texttt{LINE} is used.

It also supports the specification of a simple format string that can be
used to create a generic parser to populate additional fields of the
data item read from the stream.

The parser format is:

\begin{verbatim}
  %(IP) [%(DATE)] "%(URL)"
\end{verbatim}
This will create a parser that is able to read line in the format

\begin{verbatim}
  127.0.0.1 [2012/03/14 12:03:48 +0100] "http://example.com/index.html"
\end{verbatim}
The outcoming data item will have the attribute \texttt{IP} set to
\texttt{127.0.0.1} and the \texttt{DATE} attribute set to
\texttt{2012/03/14 12:03:48 +0100}. The \texttt{URL} attribute will be
set to \texttt{http://example.com/index.html}. In addition, the
\texttt{LINE} attribute will contain the complete line string.

\begin{figure}[h]
\begin{center}{\footnotesize
{\renewcommand{\arraystretch}{1.4}
\textsf{
\begin{tabular}{|c|c|p{9cm}|c|} \hline
\textbf{Parameter} & \textbf{Type} & \textbf{Description} & \textbf{Required} \\ \hline  
{\ttfamily key } & String &  & false\\ \hline
{\ttfamily format } & String & The format how to parse each line. Elements like {\ttfamily \%(KEY)} will be detected and automatically populated in the resulting items. & false\\ \hline
{\ttfamily id } & String &  & ? \\ \hline
{\ttfamily password } & String & The password for the stream URL (see username parameter) & false\\ \hline
{\ttfamily prefix } & String & An optional prefix string to prepend to all attribute names & false\\ \hline
{\ttfamily limit } & Long & The maximum number of items that this stream should deliver & false\\ \hline
{\ttfamily username } & String & The username required to connect to the stream URL (e.g web-user, database user) & false\\ \hline
\end{tabular}
 } 
 } 
 } 
\caption{Parameters of class {\ttfamily stream.io.LineStream}}
\end{center}
\end{figure}
