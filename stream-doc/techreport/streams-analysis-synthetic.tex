\subsection{\label{sec:syntheticData}Synthetic Data Stream Generator}
Testing online algorithms often requires a large amount of data that
matches a known distribution or can be designed such that specific
test-cases can be created for algorithms.

The \streams core package already defines a set of streams for random
data generation. In Combination with the concept of MultiStreams
this can easily be used to create tailored data streams.


\subsubsection{Example: A Gaussian Stream}
The {\ttfamily stream.generator.GaussianStream} class implements a
data stream that generates an unlimited sequence of normal distributed
data. The default setup focuses on a single attribute with a mean of
0.0 and a standard deviation of 1.0:

\begin{figure}[h!]
  \centering
  \begin{lstlisting}[language=XML]
     <stream id="gauss" class="stream.generator.GaussianStream" />
  \end{lstlisting}
\end{figure}

Using the {\ttfamily attributes} parameter allows to specify the mean and
standard deviation of one or more attributes:

\begin{figure}[h!]
  \centering
  \begin{lstlisting}[language=XML]
     <stream id="gauss-2" class="stream.generator.GaussianStream"
             attributes="0.0,1.0,2.0,0.25,8.5,2.75" />
  \end{lstlisting}
\end{figure}

The {\ttfamily gauss-2} stream above produces a sequence of data items
each of which holds attributes {\ttfamily x1}, {\ttfamily x2} and
{\ttfamily x3} based on the following distributions:

\begin{table}[h!]
  \centering
  \begin{tabular}{c|c|c} \hline
    {\bf Attribute} & {\bf Mean} & {\bf Standard Deviation} \\ \hline \hline
    {\ttfamily x1} & 0.0 & 1.0 \\ \hline
    {\ttfamily x2} & 2.0 & 0.25 \\ \hline
    {\ttfamily x3} & 8.5 & 2.75 \\ \hline
  \end{tabular}
\end{table}

The attributes are named {\ttfamily x1}, {\ttfamily x2} and {\ttfamily
  x3} but can be named according to a preset using the {\ttfamily keys}
parameter of the {\ttfamily GaussianStream} class:

\begin{figure}[h!]
  \centering
  \begin{lstlisting}[language=XML]
     <stream id="gauss-2" class="stream.generator.GaussianStream"
             attributes="0.0,1.0,2.0,0.25,8.5,2.75"
             keys="A,B,C" />
  \end{lstlisting}
\end{figure}

\subsubsection{Example: A cluster data-stream}
The stream {\ttfamily gauss-2} from above will create a sequence of
data items which are centered around (0.0,2.0,8.5) in a 3-dimensional
vector space.

By combining the concept of {\em Multistreams} with the gaussian
streams, we can easily define a stream that has multiple clusters with
pre-defined centers. The {\ttfamily RandomMultiStream} class is of big
use, here: It allows for randomly picking a substream upon reading
each item. The picks are uniformly distributed over all substreams.

The following definition specifies a stream with data items of 4
clusters with cluster centers (0.0,0.0), (1.0,1.0), (2.0,2.0) and
(3.0,3.0):

\begin{figure}[h!]
  \centering
  \begin{lstlisting}[language=XML]
    <stream id="clusters" class="stream.io.multi.RandomMultiStream">

        <stream id="cluster-1" class="stream.generator.GaussianStream"
                attributes="1.0,0.0,1.0,0.0" />

        <stream id="cluster-2" class="stream.generator.GaussianStream"
                attributes="2.0,0.0,2.0,0.0" />

        <stream id="cluster-3" class="stream.generator.GaussianStream"
                attributes="3.0,0.0,3.0,0.0" />

        <stream id="cluster-4" class="stream.generator.GaussianStream"
                attributes="4.0,0.0,4.0,0.0" />
    </stream>
  \end{lstlisting}
\end{figure}

\subsubsection{Imbalanced Distributions}
In some cases a unified distribution among the sub-streams is not what
is required. The {\ttfamily weights} parameters lets you define a
weight for each substream, resulting in a finer control of the
stream. As an example, the {\ttfamily weights} parameter can be used
to create a stream with a slight fraction of outlier data items:

\begin{figure}[h!]
  \centering
  \begin{lstlisting}[language=XML]
    <stream id="myStream" class="stream.io.multi.RandomMultiStream"
            weights="0.99,0.01">

        <stream id="normal" class="stream.generator.GaussianStream"
                attributes="1.0,0.0,1.0,0.0" />

        <stream id="outlier" class="stream.generator.GaussianStream"
                attributes="2.0,0.0,2.0,0.0" />
    </stream>
  \end{lstlisting}
\end{figure}
In this example, approximately 1\% of the data items is drawn from the
outlier stream, whereas the majority is picked from the ``normal''
stream.
