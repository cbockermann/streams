%%
%%
%%
\section{\label{sec:extendingStreams}Extending the \streams Framework}
In the previous sections we outlined how to create data flow graphs for
data stream processing by means of the XML elements that are provided
by the \streams framework. We also introduced the {\em processor}s as
the atomic functional units that provide the work required to do the
data processing.

The core \streams framework provides a rich set of basic processors that
can be used to aggregate statistics, manipulate data and even incorporates
existing libraries such as the {\em MOA} online learning library (see
Section \ref{sec:moa} for details on using MOA with \streams).

In many application use cases, we still face the problem of requiring
application specific pre-processing or functionality that cannot directly be 
achieved with the existing \streams core processors. Such functionality
can easily be added by custom implementations of processors. The \streams
framework provides a very simply Java API which encapsultes the abstract
concepts outlined in Section \ref{sec:abstraction}.

\medskip

In this section, we will highlight the general ideas of the \streams
programming API and provide a walk-through on how to implement custom
processors using the Java language. 

As pointed out earlier, the \streams framework also provides support
for scripting languages, such as JavaScript, Ruby or Python. In Section
\ref{sec:scripting} we will give an overview on how to implement custom
processors using such scripting languages.



\subsection{\label{sec:customProcessors}Implementing Custom Processors}
Processors in the {\em streams} framework can be plugged into the
processing chain to perform a series of operations on the data.  A
processor is a simple element of work that is executed for each data
item. Essentially it is a simple function:

\begin{figure}[h!]
   \centering 
   \begin{lstlisting}[language=Java]
         public Data process( Data item ){
              // your code here
              return item;
         }
   \end{lstlisting}
\end{figure}

The notion of a processor is captured by the Java interface {\ttfamily
stream.Processor} that simply defines the {\ttfamily process(Data)}
function mentioned above:

\begin{figure}[h!]
  \centering
  \begin{lstlisting}[language=Java]
      public interface Processor {
          public Data process( Data item );
      }
  \end{lstlisting}
  \caption{\label{fig:processorInterface}The interface that all processors need to implement.}
\end{figure}

Another property required for processors is that they need to provide
a {\em no-args} constructor, i.e. they need to have a constructor that
comes with no arguments.

For a wide range of common preprocessing tasks, this simple method is
sufficient enough to handle data. The processors might also maintain a
state over consecutive calls to the {\ttfamily process(Data)} method.
This {\ttfamily process} method will be called from within a single
thread only.

If a processor requires a more sophisticated configuration, e.g. for
initializing a database connection at startup or release a file handle
at shutdown, the {\ttfamily StatefulProcessor} interface can be used.
In addition to the simple {\ttfamily Processor} interface, the stateful
version adds two additional methods:
\begin{figure}[h!]
  \centering
  \begin{lstlisting}[language=Java]
    public interface StatefulProcessor extends Processor {
       /**
        *  Initialize data structures, open connections,...
        */
       public void init(ProcessContext ctx) throws Exception;

       /**
        *  Close connections, release resources,...
        */
       public void finish() throws Exception;
    }
  \end{lstlisting}
  \caption{\label{fig:statefulProcessor} The additional methods of stateful processors.}
\end{figure}

%\begin{itemize}
%  \item {\ttfamily init(ProcessContext)}
%  \item {\ttfamily finish()}
%\end{itemize}
%These methods are provide a life-cycle to the processor instances.

\subsubsection{\label{sec:processorLifecycle}The Lifecycle of a Processor}
As stated above, a processor is expected to follow some basic
conventions of the JavaBeans specification. It is expected to provide a
constructor with no arguments and should provide access to attributes
that are intended to be configurable via the XML configuration by
providing {\ttfamily set}- and {\ttfamily get}-methods.

The general life-cycle of a processor that has been added to a data flow
graph is as follows:

\begin{enumerate}
  \item An object of the processor class is being instantiated at
    container startup time.
  \item The parameters found as the XML attributes are used to
    call any {\ttfamily set}-methods that match the attribute names.
  \item If the processor class implements the {\ttfamily StatefulProcessor}
    interface, the {\ttfamily init(ProcessContext)} method will be called.
  \item The {\ttfamily process(Data)} method is called for all data items
    that the parent process of the processor receives.
  \item As the container shuts down, the {\ttfamily finish()} method of
    the processor is called {\em if} the processor class implements the
    {\ttfamily StatefulProcessor} interface.
\end{enumerate}




\subsubsection{\label{sec:exampleProcessor}Example: A simple custom processor}
In the following, we will walk through a very simple example to show
the implementation of a processor in more detail. We will start with a
basic class and extend this to have a complete processor in the end.

The main construct is a Java class within a package {\ttfamily my.package}
that implements the identity function is given as:

\begin{figure}[h!]
  \centering
  \begin{lstlisting}[language=Java]
      package my.package;

      public class Multiplier implements Processor {
           public Data process( Data item ){
               return item;
           }
      }
  \end{lstlisting}
\end{figure}

This class implements a processor that simply passes through each data
item to be further processed by all subsequent processors.  Once
compiled, this simple processor is ready to be used within a simple
stream processing chain. To use it, we can directly use the XML syntax
of the \streams framework to include it in to the process:

\begin{figure}[h!]
  \centering {\footnotesize
  \begin{lstlisting}[language=XML]
       <container>
         <process input="...">
              <!-- simply add an XML element for the new processor   -->
              <my.package.Multiplier />
         </process>
       </container>
  \end{lstlisting}}
\caption{\label{fig:multiplyXML}The processors are added to the XML
  process definition by simply adding an XML element with the name of
  the implementing class into the process that should contain the
  processor.}
\end{figure}


\subsubsection*{Processing data}
The simple example shows the direct correspondence between the XML
definition of a container and the associated Java implemented
processors. The data items are represented as simple Hashmaps with
{\ttfamily String} keys and {\ttfamily Serializable} values.

The code in Figure \ref{fig:multiplyImpl} extends the empty data
processor from above by checking for the attribute with key {\ttfamily
x} and adding a new attribute with key {\ttfamily y} by multiplying
{\ttfamily x} by 2. This simple multiplier relies on parsing the double value from its
string representation. If the double is available as Double object
already in the item, then we could also directly cast the value into a
Double:

\begin{figure}[h!]
  \centering{\footnotesize
  \begin{lstlisting}[language=Java]
     // directly cast the serializable value to a Double object:
     Double x = (Double) item.get( "x" );
  \end{lstlisting}}
\end{figure}

The multiplier will be created at the startup of the experiment and
will be called (i.e. the {\ttfamily process(..)} method) for each
event of the data stream.

\begin{figure}[h!]
  \centering{\footnotesize
  \begin{lstlisting}[language=Java]
     package my.package;
     import stream.*;

     public class Multiplier implements Processor {
         public Data process( Data item ){
             Serializable value = item.get( "x" );	     

             if( value != null  ){
                Double x = new Double( value.toString() );  // parse value to double
                data.put( "y",  new Double(  2 * x ) );     // multiply+add result
             }
             return item;
         }
     }
  \end{lstlisting}}
\caption{\label{fig:multiplyImpl}A simple custom processor that
  multiplies an attribute {\ttfamily x} in each data item by a
  constant factor of 2. If the attribute {\ttfamily x} is not present,
  this processor will leave the data item unchanged.}
\end{figure}


\subsubsection{Adding Parameters to Processors}
In most cases, we want to add a simple method for parameterizing our
Processor implementation. This can easily be done by following the
{\em Convention-over-Configuration} paradigm: By convention, all
{\ttfamily setX(...)} and {\ttfamily getY()} methods are automatically
regarded as parameters for the data processors and directly available
as XML attributes.

In the example from above, we want to add two parameters: {\ttfamily
key} and {\ttfamily factor} to our Multiplier implementation. The
{\ttfamily key} parameter will be used to select the attribute used
instead of {\ttfamily x} and the {\ttfamily factor} will be a value
used for multiplying (instead of the constant {\ttfamily 2} as above).

To add these two parameters to our Multiplier, we only need to provide
corresponding getters and setters as shown in Figure
\ref{fig:multiplyParameters}.
        
After compiling this class, we can directly use the new parameters
{\ttfamily key} and {\ttfamily factor} as XML attributes. For example,
to multiply all attributes {\ttfamily z} by {\ttfamily 3.1415}, we can
use the following XML setup:

\begin{figure}[h!]
  \centering
  \begin{lstlisting}[language=XML]
       <container>
           ...
           <process input="...">
               <my.package.Multiplier key="z" factor="3.1415" />
            </process>
       </container>
  \end{lstlisting}
  \caption{\label{fig:multiplyParametersXML}}
\end{figure}

Upon startup, the getters and setters of the Multiplier class will be
checked and if the argument is a Double (or Integer, Float,...) it
will be automatically converted to that type.

In the example of our extended Multiplier, the {\ttfamily factor}
parameter will be created to a Double object of value {\ttfamily 3.1415} and
used as argument in the {\ttfamily setFactor(..)} method.


\begin{figure}[h!]
  \centering \footnotesize{
  \begin{lstlisting}[language=Java]
     // imports left out for truncation
     //
     public class Multiplier implements Processor {
        String key = "x";    // by default we still use 'x'
        Double factor = 2;   // by default we multiply with 2

        // getter/setter for parameter "key"
        //
        public void setKey( String key ){
            this.key = key;
        }

        public String getKey()(
            return key;
        }

        // getter/setter for parameter "factor"
        // 
        public void setFactor( Double fact ){
            this.factor = fact;
        }

        public Double getFactor(){
            return factor;
        }
     }
  \end{lstlisting}}
  \caption{\label{fig:multiplyParameters}The {\ttfamily Multiplier} processor with added parameters.}
\end{figure}

\subsection{\label{sec:customServices}Adding custom Services}
\subsection{\label{sec:scripting}Using Scripting Languages in \streams}
Scripting languages provide a convenient way to integrate ad-hoc
functionality into stream processes. Based on the Java Scripting
Engine that is provided within the Java virtual machine, the streams
library includes support for several scripting languages, most notably
the JavaScript language.

Additional scripting languages are being supported by the
ScriptingEngine interfaces of the Java virtual machine. This requires
the corresponding Java implementations (Java archives) to be available
on the classpath when starting the \streams runtime.

Currently the following scripting languages are supported:
\begin{itemize}
   \item JavaScript (built into the Java VM)
   \item JRuby (requires jruby-library in classpath).
\end{itemize}
Further support for integrating additional languages like Python is
planned.\baustelle


\subsubsection{\label{sec:javascript}Using JavaScript for Processing}
The JavaScript language has been part of the Java API for some
time. The \streams framework provides a simple {\ttfamily JavaScript}
processor, that can be used to run JavaScript functions on data items
as shown in Figure \ref{fig:javascriptExample}.

\begin{figure}[h!]
  \centering
  \begin{lstlisting}[language=XML]
       <container>
          ...
          <process input="...">
              <!--  Execute a process(data) function defined in the
                    specified JavaScript file     -->
              <JavaScript file="/path/to/myScript.js" />
          </process>
       </container>
  \end{lstlisting}
  \caption{\label{fig:javascriptExample}The {\ttfamily JavaScript} processor applies process() functions defined in JavaScript.}
\end{figure}


Within the JavaScript environment, the data items are accessible at
{\ttfamily data}. Figure \ref{fig:javascriptProcessor} shows an
example for the JavaScript code which implements a processor within
the file {\ttfamily myScript.js}.

\begin{figure}[h!]
  \centering
  \begin{lstlisting}[language=JavaScript,showspaces=false,showstringspaces=false]
       function process(data){
          var id = data.get( "@id" );
          if( id != null ){
             println( "ID of item is: " + id );
          }
          return data;
        }
  \end{lstlisting}
  \caption{\label{fig:javascriptProcessor}JavaScript code that implements a processor.}
\end{figure}